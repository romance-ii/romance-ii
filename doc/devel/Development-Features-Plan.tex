% Created 2013-11-21 Thu 18:06
\documentclass[11pt]{article}
\usepackage[utf8]{inputenc}
\usepackage[T1]{fontenc}
\usepackage{fixltx2e}
\usepackage{graphicx}
\usepackage{longtable}
\usepackage{float}
\usepackage{wrapfig}
\usepackage{rotating}
\usepackage[normalem]{ulem}
\usepackage{amsmath}
\usepackage{textcomp}
\usepackage{marvosym}
\usepackage{wasysym}
\usepackage{amssymb}
\usepackage{hyperref}
\tolerance=1000
\author{Bruce-Robert Fenn Pocock}
\date{\today}
\title{Development-Features-Plan}
\hypersetup{
  pdfkeywords={},
  pdfsubject={},
  pdfcreator={Emacs 24.3.1 (Org mode 8.2.3c)}}
\begin{document}

\maketitle
\tableofcontents

\section{Violet Volts: Overview}
\label{sec-1}
\subsection{Manifesto}
\label{sec-1-1}

The concept I'm running off of goes a bit like this

\subsubsection{Axiom: Violence is filler in most games.}
\label{sec-1-1-1}

EG: Zelda is mostly about solving the puzzles, and while a few of them
(shoot  the boss  in  the  eye) are  couched  themselves  in terms  of
violence,  the  majority  of  the   violence  in  the  game  world  is
filler. Killing Bokoblins to maybe get a Rupee can be kinda fun, sure,
but it's not moving the plot forward or really providing much “puzzle”
content, \&c \&c.
\subsubsection{Hypothesis: Non-violence can be fun.}
\label{sec-1-1-2}

OTOH, non-violent games tend to be of a very few sorts;

\begin{itemize}
\item Some, like Pac-Man = Katamari Damacy, are basically aiming to
“clear the board” in some way. Alchemy from Pop Cap is another of
this sort.

\item Others, like Bejeweled or Tetris, are basically “endless.” You just
repeat the same task over and over, until you lose.

\begin{itemize}
\item The alternative are games in which the designers have spent a great
deal of time creating a puzzle — which you can probably (a) solve
faster than they can create new ones; (b) look up the solution on
the 'Net in 45 seconds or less.
\end{itemize}
\end{itemize}

All of  these put the idea  of a non-violent MMO-RPG  in a persistent,
shared world at risk.

\begin{itemize}
\item We  can't use violence as  filler.

\item We can't let someone “clear the board,” because the next person
will have nothing to do. 

\begin{itemize}
\item We can, but shouldn't, make it an endless toil: that, eventually,
isn't fun.

\item We can't afford the time effort to have more designers than players
to create specific puzzles.
\end{itemize}
\end{itemize}

So, the sandbox world has these things going for it:

\begin{itemize}
\item User generated content: Building and decorating houses and
landscaping your yard, designing and wearing clothes, and so
forth.

\item Crafting and similar rôte tasks; these are like Bejeweled, but you
only do it when you want the results

\item AI's who have their own wants/needs/likes. They provide Quest
drivers, basically. The AI's can “intentionally” create situations
which are puzzles to be solved.
\end{itemize}

EG:  Some “naughty”  AI character  likes to  build mazes  and has  the
ability  to  create  a  certain   type  of  item  that  is  reasonably
valuable. Solving the maze gives a  reward; but the AI re-arranges the
maze from  time to time,  so you can re-visit  it later, and  it'll be
different. We  started to build  something like this in  Tootsville 3,
but the  front-end choked  on the  number of  “furniture” items  to be
placed on the screen.

\subsubsection{About Animal Crossing}
\label{sec-1-1-3}

Animal Crossing is  really more about:

\begin{itemize}
\item meeting and interacting with the characters, making friends

\item creating User Generated Content: decorate your house, design your
clothes, decorate signs, change the Town Song, …

\item discovering and collecting artefacts: shells, fruit, fossils, … ;
sometimes these can be traded

\item special events [The jazz playing dog on Saturday nights; the Turnip
“stock exchange”; holiday events; special visiting characters]
\end{itemize}

In  MMO  context, you  can 

\begin{itemize}
\item meet and interact with AI and human controlled characters, make
friends

\item create User Generate Content of many of the same kinds

\item explore new regions of the world as they are added

\item collect different kinds of artefacts and trade them with other
characters

\item special events of similar kinds
\end{itemize}
\subsubsection{About World Rasslin Federation}
\label{sec-1-1-4}

Chris Brunner  brought us the  world of WWF Tootsville.  AI characters
get controlled by staff members to act  out a story for the players to
watch / join in. This takes it further: we give the AI's programs that
inherently create “fun” situations when they interact.
\subsubsection{About in-world minigames}
\label{sec-1-1-5}

In Tootsville  4, we had  a (working)  soccer park. [I  think actually
that Meixell  had tried to make  something like this in  Tootsville 2,
but it didn't work because the  roomVariable system in SFS didn't work
right.] The  Referee (Stu) was a  robot that kept score  [and reset it
after a while].  The players “kicked” the ball by  walking into it. If
it rolled into a  goal, they got a point; if it  rolled out of bounds,
the Referee would walk over and put it at the bounds. The rules of the
game were otherwise unenforced. The kids still played the game. We got
more complaints about  people “cheating” by buying things  in the Mall
than about the soccer game.

Putting open-ended things like this in  the game world might be enough
to  amuse some  people. Perhaps  a “mission”  or “greater  purpose” is
needed  only  to  give you  an  excuse  to  explore  and play  in  the
world. How often have we ignored Navi so we could play some other game
and ignore our Great Quest?
\subsubsection{Miyamoto is my Hero}
\label{sec-1-1-6}

Miyamoto-san:

How do I want gamers to play Super Mario Sunshine? In each Mario game,
players sometimes move  around without any specific  purpose, yet they
may find some secret as a  result. While wandering around, players get
accustomed to the  gameplay so that Mario moves just  as they want him
to. Then  gamers feel some  attachment to the  game and don't  want to
sell  it to  used-software  shops. So,  the more  you  play, the  more
attachment  you  feel.  This  is  an important  element  we  want  you
to advertise.

(Will Wright on  Miyamoto) Wright had said, “When you  play his games,
you feel like you’re a kid and  you’re out in the back yard playing in
the dirt.”
\subsubsection{Caillois}
\label{sec-1-1-7}

Caillois divides  play into four categories:  agon [competition], alea
[chance], mimicry [simulation], and ilinx [vertigo].
\subsubsection{The Great Experiment}
\label{sec-1-1-8}

Altogether it's mostly an experiment: can we use AI characters to help
the players create their own stories that aren't “a war?”

\begin{itemize}
\item Eve  has shown  that players  can organize  themselves to  create a
story  built  around  war;  but  we  can  enforce  a  Shapir-Whorff
hypothesis in a game world. You lack the vocabulary in the language
of  actions  presented  to  you  as a  player  to  conduct  literal
violence.  There  are no  swords.  So,  will players  devolve  into
bullies in  some other way, and  end up creating their  own war? Or
will  they  find  a  way  to tell  more  interesting  stories  with
their play?

\item So  many stories we  do tell  are simple and  non-violent. Romantic
comedies.  Sitcoms.  Even  horror  classics  are  not  “about”  the
violence;  even   in  action  films,   the  violence  is   often  a
filler. Given a vocabulary of  activities that tends to push toward
creativity and  forbids violence, will players  use that creativity
and keep coming back? or will they feel like “what this game really
needs is to let me kill the ogres?”
\end{itemize}
\subsubsection{The danger}
\label{sec-1-1-9}

Nonetheless, there is a certain amount of “danger” necessary.

Having someone trash your garden because you didn't stop him from
coming into your neighbourhood, or ogres steal your furniture, or
something making you unable to do things in the world are forms of
“hurting” you that mean you have consequences of your actions, but you
can't just “shoot him with a BFG-2000,” you need to find a way to
“trick” him or convince him to somehow be beneficial.
\subsubsection{Major Point: Sidekicks}
\label{sec-1-1-10}

The Sidekick model (bring along a player who hasn't registered
themself; they share your inventory) might help lure more players in,
particularly with the low barrier to start playing (Click this link in
Firefox, and you're in the game now)

With some  luck, that means  we could grow  the player base  better as
people  use their  existing social  networks [literal  or digital]  to
expose  the game  to more  people, particularly  kids. Tootsville  4's
registration took about 4 minutes. This  brings it down to almost zero
effort to join  in, if you know  an existing player. It  also makes me
hope that  most new players  will see the  benefits of having  a guide
with them, so we won't have to worry so much about tutorial needs.

Bonus: marketing: “play the game with your kids” and so forth.

Analogy:  “girlfriend   mode”  was  the  nickname   for  options  like
\url{http://zeldawiki.org/Tingle_Tuner}  which allowed  a  second player  to
“help” without having to “commit” to playing an 80-hour-long RPG.
\subsubsection{Major Point: Persistent World}
\label{sec-1-1-11}

No instanced areas. Very little Adamantium construction.
\subsubsection{Major Failure: Monetization}
\label{sec-1-1-12}

I have no solid idea how we could hope to make any money doing this.

\begin{itemize}
\item Selling items? But then, can't people cheat by buying things for
USD\$ rather than earning them in game?

\begin{itemize}
\item Paying to clone an item that you have earned, e.g. if you have
multiple characters? to give to a friend?

\item Cosmetic items only?  Charge them 50¢ to dye their clothing?
\end{itemize}

\item Subscription? :-P I doubt that could be a good idea.

\begin{itemize}
\item Hibernation charge?  Protect your  character from entropy. Pay us
\$1 to freeze time when you're not logged in.  No need to mow your
lawn. But: it breaks the  shared, persistent world model. Need an
in-game excuse for this to work.
\end{itemize}

\item Merchandising
\end{itemize}

There's no profit model

OTOH, the hosting  costs should be pretty light and  it's a fun hobby,
so  I'm  willing to  dump  a  certain amount  towards  it  for my  own
amusement \&  my friends'; and,  maybe we  can sell enough  manuals and
coffee mugs on Cafe Press to pay the hosting bill.

There's not  much of a  marketing model,  which is why  Tootsville 1-4
ultimately failed. “Word of mouth” takes longer than “pay your hosting
bill” will accept.  The ideas of the Manifesto might  be unique enough
that we could  luck out and get Slashdotted by  someone like Upworthy,
but I wouldn't want to count on it.

Without solving these  two, I wouldn't want to  accept any investment,
meagre or  otherwise, because  it's being  built, realistically,  as a
money pit.

Perhaps getting someone to pay me (us)  to do this over again for them
is the real profit model.

Louis        wants         to        build         these        things
\url{http://resinteractive.com/portfolio.html}  but who  would give  him the
money with that track record?

Maybe we get a better royalty split and a better revenue model (more
adult players or more affluent players) and create something that's
more marketable/profitable.
\subsection{The Team}
\label{sec-1-2}

\subsubsection{Me! Bruce-Robert Fenn Pocock}
\label{sec-1-2-1}

designer and programmer and stuff. whatever is needed 

Res co-owner
\subsubsection{Gene Cronk}
\label{sec-1-2-2}

has volunteered to help with some sysop tasks (at least)
\subsubsection{Erick Feiling}
\label{sec-1-2-3}

may be interested in some hacking
\subsubsection{Joseph Williamson}
\label{sec-1-2-4}

may be interested in some hacking
\subsubsection{Mark Mc Corkle}
\label{sec-1-2-5}

may be interested in some hacking \& has good industry advice on 

Res co-owner
\subsubsection{Chris Brunner}
\label{sec-1-2-6}

moral support

also a Res co-owner

\subsubsection{Testers}
\label{sec-1-2-7}

Kittie,  Sage, Jess,  James,  Carly  + kids,  Meredith  + kids,  Megan
Griffith, 
\subsection{Rough plans}
\label{sec-1-3}

Violet Volts,  of course, is an  anagram for Tootsville V; so that
name has got to go, because Louis decided to piss himself. Not
shocking, but sad.

\subsubsection{No Tootsville?}
\label{sec-1-3-1}

No Tootsville.

\begin{enumerate}
\item Plan B: Tortoises
\label{sec-1-3-1-1}

Conversation with Carly Robb:

BRFP but, like I said, this is my hobby project, so I might as well
     resurrect the thing.  and if not, maybe we'll use owls.
     anybody using owls?

CR hehe, you never know\ldots{}. not that i know of\ldots{}  owls are good\ldots{}

BRFP or something else that's easy to draw :-) I like owls.
     Big hats. Funny glasses.

CR tortoises are nice too\ldots{}.

BRPF Hmm.  true paint their shells difficult to dress as a spaceman?
     I guess bipedal.

CR and they're semi-elephantish with the stumpy legs\ldots{}  just give 'em
     a helmet, they'll do fine\ldots{}.

BRFP no real elbows to speak of.  makes animation easier.

CR hehe
\end{enumerate}

\subsubsection{{\bfseries\sffamily DONE} awaiting answer from Louis Pecci\hfill{}\textsc{RES}}
\label{sec-1-3-2}

Louis would “never” consider a free game offer. Oh, well, fuck him.

\begin{enumerate}
\item {\bfseries\sffamily DONE} nag Louis a little\hfill{}\textsc{RES}
\label{sec-1-3-2-1}
\end{enumerate}
\subsubsection{{\bfseries\sffamily TODO} set up fucking Bugzilla and log tickets for all this shit}
\label{sec-1-3-3}
instead of having it in a text file

\begin{enumerate}
\item Maybe Gene would “like” to do this ?\hfill{}\textsc{GENE}
\label{sec-1-3-3-1}
\end{enumerate}
\subsection{Core Design}
\label{sec-1-4}

Romance Ⅱ server core. New  message queue based system, replacing the
packet system from Quaestor; binary compiled Lisp, so much faster than
Java; lots of nice stuff.

I'm writing up a  rather exhaustive manual for that. I  hope to sell a
few copies of the book. I'm even  trying to write some of the chapters
up  like  academic papers;  I  hope  to circulate  them  individually,
as well.

The  real  reason  for  the  great manual,  though,  is  to  get  more
contributors.  If  some  companies  start using  this  for  their  own
MMO-RPG's, then, yay.
\subsection{Front-end: HTML5, pseudo-3D.}
\label{sec-1-5}

\subsubsection{Partial 3D: “Diorama” style}
\label{sec-1-5-1}

The  world itself  will  be largely  3D polygonal,  but  with lots  of
"billboard" elements.  The opposite of  how FF-VII did it,  where they
had painted backgrounds and 3D characters.

If I  have to create new  avatars, they might be  low-poly 3D instead,
and fuck the 2D part.

The camera is permanently fixed looking  down, and to the "north." So,
we never see the north-facing side  of anything, ever. This also means
that the billboarded graphics won't look as silly.

If I  could afford to recreate  everything in "real" 3D,  I'd consider
it, but this seems like the  best "cheat" for now, and the performance
should be  good even on  (for example)  cheap Android tablets.  But, I
might have to  use a native (Java or binary)  client to get reasonable
3D on them. Not sure yet.

\begin{enumerate}
\item {\bfseries\sffamily TODO} Consider buying a tablet
\label{sec-1-5-1-1}

like Ginny's kids have for testing.

Some kind of 7-10" Android tab
\item {\bfseries\sffamily TODO} 3D library
\label{sec-1-5-1-2}

I don't  give a shit  if it  works in MSIE,  although I guess  MSIE 10
might be worth looking into someday.

Want something for WebKit + Blink + Gecko though.

WebGL is still kinda shitty. Sad panda.

How good can I get without

\begin{itemize}
\item needing to spend a year in Blender to get shit done;
\item requiring players to buy a \$3,500 PC and reconfigure their video
card to get the thing to load
\item taking advanced math classes to figure out WTF is going on in the
engine's API
\end{itemize}

\begin{enumerate}
\item GEGL
\label{sec-1-5-1-2-1}

So far, this looks like a winner.

Not “married” to it, but it looks pretty good.
\item {\bfseries\sffamily TODO} MSIE WebGL plug-in\hfill{}\textsc{MSIE:HELP}
\label{sec-1-5-1-2-2}

I set up something that might help people install the plug-in, but
haven't really tested it
\item {\bfseries\sffamily TODO} Android/Chrome enabler help\hfill{}\textsc{HELP:ANDROID:CHROME}
\label{sec-1-5-1-2-3}

\item {\bfseries\sffamily DONE} Opera desktop enabler help\hfill{}\textsc{HELP:OPERA}
\label{sec-1-5-1-2-4}
\end{enumerate}
\end{enumerate}
\subsubsection{UI Stuff}
\label{sec-1-5-2}

The UI is JUST A UI. There's no real game-logic on the UI side, except
for knowing how to interact with the user.

If  I get  a  chance to  rewrite  it for  other  platforms, then,  all
the better.

\begin{enumerate}
\item Basic Screen layout
\label{sec-1-5-2-1}

The UI  is all one-finger; we  look for taps/clicks, and  drags. Mouse
buttons 2 \& 3 should be ignored for now (this includes trapping button
3 so it can't throw up a context menu)


In the four corners of the screen we have the four main controls:

\begin{itemize}
\item equipped item;  tap to  "use" it;  what that  means depends  on the
item; also mapped to the Spacebar

\item a "status"  icon to bring  up various things,  including inventory;
also mapped to the Escape key and Tab key.

Probably  derived from  the equipment  icon but  with some  kind of
player-avatar representation in it.

\item a "face/step"  control. Tap  once to  turn to  face a  direction (8
ways); tap when already facing that way to take 1 step. Also mapped
to  the  cursor   keys  and  the  numeric   keypad  (digits  except
5). Basically meant  to look like a gamepad stick  and be easier to
“tap”   to  move   in   narrow  spaces   or  turn-to-face   without
actually walking.

\item "Action"  control. Depends  on  context. Think  the  (A) button  in
Ocarina  of Time.  Also mapped  to  Enter, Return,  or the  numeric
keypad 5  key. A small "side"  icon (which jumps up  and takes over
when you're near someone) is for "say something" all the time. (See
"tap and talk")
\end{itemize}


For a11y we  may need/want to be  able to cluster these in  one or two
corners,   also.  I   have  some   ideas  about   giving  pre-arranged
re-arrangements for  this for  users who  (a) are  using a  tablet and
holding it from  the bottom corners (thumby controls) and  (b) who are
using some kind of pointer (say,  a trackball) where it's difficult to
move  across  the  width  of  the   screen,  so  they  want  them  all
combined together.

Tap on the ground to move to a position, using pathfinding.
\item Items and stuff
\label{sec-1-5-2-2}

"Usable  items" ---  you can  only  equip one  item to  the button  at
a time.

Items  can have  continuous (percentage-full)  or discrete  (countable
number)  quantities associated  with them;  both  have a  "max" and  a
"current" amount. This was done in Romance 1.1 as well.

Items can be "targeted," "directed," or "directional," or "immediate."

TARGETED:  Click  on   a  specific  entity  to   affect.  Click  item;
click entity.  Entity should “highlight”  when mouse over  to indicate
that it's eligible for selection. Gamepad: L-target then “Fire”

DIRECTED: Click on a coördinate; tries  to affect that point in space,
regardless  of  whether  there  is   any  entity  there.  Click  item;
click  space.  Cursor  should  be  a “ball”  of  some  kind.  Gamepad:
R-stick aim? 

DIRECTIONAL:  Click   to  set  a  direction,   relative  to  yourself;
e.g.  click up  top-left when  standing at  center screen  means, "aim
north-westerly"  Click  item,  then   position  pointer  “around”  the
character; should  maybe show a  vector moving parallel to  the ground
plane in  the chosen direction  toward infinity to indicate  this mode
of selection. Gamepad: R-stick aim?

IMMEDIATE: Click to use, here, now. No second click.
\item {\bfseries\sffamily TODO} Gamepad support?\hfill{}\textsc{W3C:CHROME}
\label{sec-1-5-2-3}

There  looks   to  be  a   Gamepad  support  for   HTML5  supplemental
recommendation,  but I  don't see  any  signs that  anybody is  really
supporting it yet?
\end{enumerate}
\subsubsection{Server communication}
\label{sec-1-5-3}

streaming, probably WebSockets; server should accept raw TCP too.

Entities get tracks;  maybe we can use the CSS  animation functions in
some  cases, which  is  nifty. (*no,  we can't,  in  WebGL things  are
different)

Aggregate  entities  (e.g.   character  +  costumes)  are  manipulated
hierarchically; so if  I move the aggregate, the  parts move ensemble.
Normal  behaviour for  OpenGL. Different  than what  I was  sending to
Persephone, though, where Osiris was handling the subspace summing for
the most part.

Basically sending  sexp's, probably want  to use the  MongoDB protocol
for the  wire protocol  this time  around, it's  pretty nice.  BSON, I
think. It's Binary JSON. Just a little tighter.
\subsection{Navigation\hfill{}\textsc{NAV:PHYSICS}}
\label{sec-1-6}

"walkable  spaces" are  totally different  in this  model, plus  we're
going to have server-side physics, so there.


"walkable"  surfaces =  defined  by interaction  of  character type  +
surface;  always  must be  a  "real"  3D  surface (not  a  billboarded
object); generally defines as  "navigable" "difficult" or "impassible"
for the character; the client shouldn't need to know this stuff.

The character will pathfind "navigable" before "difficult" usually.

eg:  "navigable" terrain  = flat  earth. "difficult"  terrain =  loose
sand, or steep incline uphill. "impassible"  might be a wall. They may
be able to jump it, but they should usually walk around.


\subsubsection{Adrenaline, Exertion}
\label{sec-1-6-1}


Character "adrenaline" v "exertion" scale similar to Skyward Sword. If
you  start going  acrobatics (e.g.  navigating difficult  terrain) the
"adrenaline" scale  goes up,  so you'll accept  the "penalty"  to your
exertion scale and  use the more difficult route (e.g.  jumping over a
small wall) rather than walking around.

\subsection{Equipment, Slots, Valences}
\label{sec-1-7}


Some equipment are clothes. All clothes are equipment.

Equipment can:

\begin{itemize}
\item have  a  passive effect;  simply  equipping  it  allows it  to  "do
something"

\item have  an active  effect; this  usually  means it  becomes the  item
mapped to your "current item" button.
\end{itemize}

Slots:  are  like  "shirt/chest"  "pants/groin"  "left  hand"  and  so
forth. Items are only equippable in certain slots.

Valences: Some  things can be stacked  within a slot. Like,  you could
maybe have a character wearing

\begin{itemize}
\item body paint (pattern layer)
\item pasties
\item bra
\item undershirt
\item shirt
\item waistcoast
\item jacket
\item overcoat
\item cape
\end{itemize}

… all of these taking the "shirt" slot. Some of them block other slots
at the same valence though. Like, the  cape might have a hoot and thus
extend into a "hat" slot. The "pattern" might cover the whole body. \&c

The avatar "exports" its slots. The item "imports" them.

Basic "DON/DOFF" commands.

Hot-swap icon  for equipment on  a "swipe?"  Hide one icon  behind the
other and "swipe" to swap them? worth considering.

In real life it takes a second to swap items, so no worries if it does
in the game, too.
\subsection{Server Model}
\label{sec-1-8}

Multiple servers running lightweight containers

Containers  win for  sharing  mmapped regions  of  RO content  between
themselves  so we  don't lose  anything  if we  run multiple  "logical
servers"  on  one  "physical"  (possibly   also  a  VM,  e.g.  Amazon)
server.  And  we  just  design  each one  to  run  in  a  dead-minimal
environment all by itself, no  worries about other services running on
the same host.

Game core itself  = monolithic executable that  sniffs its environment
and command-line  to decide which task  to do on that  instance; thus,
lots of shared  RO storage, and so  much of a game  server is overhead
logic stuff that there isn't much of a ballooning either way.


Message  Queue  =  leaning  toward ZeroMQ  or  RabbitMQ.  No  opinions
yet. Need to research. Main question, which  is easier to set up for a
dynamic  environment where  we  might spin  up  more server  instances
(containers or hosts)

DB = Postgres. Need to figure out a clustering solution.
\subsection{Dev Infrastructure\hfill{}\textsc{DEVEL}}
\label{sec-1-9}

Open source components \url{http://github.com/romance-ii}

Private stuff @ Raven:/pub/Software/ for now. Git also.

\begin{verbatim}
git clone username@elysium.star-hope.org:/pub/Software/violet-volts ~/Projects/
\end{verbatim}

\subsubsection{{\bfseries\sffamily TODO} Separate "assets" git tree.}
\label{sec-1-9-1}

\subsubsection{{\bfseries\sffamily TODO} set up Bugzilla\hfill{}\textsc{GENE}}
\label{sec-1-9-2}

Bugzilla --  I can  set up the  config, but  it's a bit  of a  pain to
build. Separate  Postgres instance  for hosting  its data,  versus the
game; or at least its own schema, if it knows how to do that. (They're
cool with pg these days, right?)
\subsubsection{{\bfseries\sffamily TODO} set up Koji\hfill{}\textsc{GENE}}
\label{sec-1-9-3}

Koji --  argh. requires some  other infrastructure  to set it  up, but
it's really what we need. If I can't get it running, I might fall back
on Jenkins instead, but ick. Infrastructure like Kerberos.
\subsubsection{{\bfseries\sffamily TODO} Postfix}
\label{sec-1-9-4}

Mail server = Postfix.

Eventually we'll want some automated mail handling stuff.
\subsubsection{{\bfseries\sffamily TODO} static web server?}
\label{sec-1-9-5}

Web (static) server  = flip a coin, nginx or  Apache? shouldn't matter
much, the "fun" stuff will all be on the dynamic server anyways

Might just say fuckall and use aserve or Hunchintoot and build it into
the same executable …
\subsection{Tap And Talk\hfill{}\textsc{TAPNTALK}}
\label{sec-1-10}

Madlibs-style interface

limited vocabulary

map directly to logical assertions or queries

separate content from language

big issue = selecting nouns. lots of stuff to choose from, how to make
it not take forever to bring up a concept?

most-recently-used set will help

but might need to "talk about"  things that are not in your inventory,
far away, whatever; that will make them ornery to select

if you haven't "heard of it," you can't talk about it.

Shapir-Whorff Hypothesis in action for the AI's too.


\subsubsection{SINGING tortoises\hfill{}\textsc{SOUND}}
\label{sec-1-10-1}

Pick a voice and we'll run the dialogue through some kind of "note
generator" so whenever you say the same word, it'll "sound"
the same.

Not English hashes like we had looked at before. I'll just take the
conceptual tokens used in the AI and use them as indices to pick a
starting and ending frequency and duration.

Maybe MOD-style. Pitch-bend, duration, and pick a "sample" to play.

\begin{enumerate}
\item {\bfseries\sffamily TODO} What kinds of sounds do tortoises make?
\label{sec-1-10-1-1}

\item {\bfseries\sffamily TODO} sound in the browser
\label{sec-1-10-1-2}

How shitty is it? Is it as bad as I think it is?

Might need to create a "sound rendering" server and stream this to
the client.

My initial tests at synthesizing music had “crackling” effects that I
couldn't shake. Need to reconsider/re-evaluate this.
\end{enumerate}
\subsection{{\bfseries\sffamily TODO} Sound effects; foley\hfill{}\textsc{SOUND}}
\label{sec-1-11}
\subsection{{\bfseries\sffamily TODO} background music\hfill{}\textsc{SOUND:MUSIC}}
\label{sec-1-12}
\subsubsection{{\bfseries\sffamily TODO} Creative Commons music}
\label{sec-1-12-1}

Look into pulling Creative Commons music en masse for the “radio”
\subsection{Geography\hfill{}\textsc{MAP}}
\label{sec-1-13}

There WILL be one big map. I'll probably run some noise generator over
a sketched map to create a huge heightmap of the island.

\subsubsection{{\bfseries\sffamily TODO} re-draw a non-Tootanga map.\hfill{}\textsc{MAP:BRP}}
\label{sec-1-13-1}
\subsection{Physics\hfill{}\textsc{PHYSICS}}
\label{sec-1-14}

Looks like it's Bullet.

NB: this is handled server-side mostly, but the client will need/want
to predict things also. Having the same library compiled two ways
would be the optimal solution, of course.

\subsubsection{{\bfseries\sffamily TODO} make new Lisp bindings, the old ones are kinda shitty}
\label{sec-1-14-1}
\subsubsection{{\bfseries\sffamily TODO} make a JavaScript binding?}
\label{sec-1-14-2}
\subsection{AI planners\hfill{}\textsc{AGENT}}
\label{sec-1-15}

A-star Djikstra style (give or take)

their "intelligence"  score is kinda  the number of steps  the planner
evaluates per "turn"

same I/O as players. only distinction P-C v NPC = where the I/O goes.

temporal associative logic

Andi-land style stuff

conceptual trees for memory: when do they forget?

when  do   they  "notice"   something?  ---  interests   trip  passive
perception; active perception triggered on other events
\subsection{Teams\hfill{}\textsc{TEAMS}}
\label{sec-1-16}

loose thing

really just a UI thing?

any net effects? maybe passive items  activate for a whole team. (mass
buff/debuff)
\subsection{Buddy List?\hfill{}\textsc{BUDDYLIST}}
\label{sec-1-17}

eh. I really don't care.

double-accept? (Facebook-style or Google+-style?)
\subsection{Sidekick\hfill{}\textsc{SIDEKICK}}
\label{sec-1-18}

Non-registered player-character shows up as "XXXX's friend"

basically treated like a pet

but  they share  the same  inventory, so  you have  to loan  them your
clothes and tools.

\begin{itemize}
\item Zero barrier to entry to play
\item Parents can bring their kids on or vice-versa
\end{itemize}
\subsection{Pets\hfill{}\textsc{PETS}}
\label{sec-1-19}

follow you around, and stuff

basically simpler AI's

non-linguistic

same as any other animal, but they "like" you a lot
\subsection{Vehicles\hfill{}\textsc{VEHICLE}}
\label{sec-1-20}

definitely want a  \textbf{high} speed rail network

also revamp the idea of cars for both transportation and racing — the
mechanic/inventor will invent them after we're online

“walkable space” concept to keep cars only on the road

“obey the crosswalk” magic gimmicks, like a teaching video; as soon as
someone enters a crosswalk, they proceed across it without running,
cars must stop.
\subsection{Passports\hfill{}\textsc{PASSPORT}}
\label{sec-1-21}

In Tootsville 2,  once you got invited  to a place, you  could go back
there by clicking on your passport  icon, but we ditched or broke that
concept pretty early on.

Now  that you  can walk  across the  whole world,  though, maybe  once
you've found  a train  station or  fountain in a  village you  can get
your passport stamped  --- perhaps even have some  minigame type thing
to get your stamp  --- and once your passport is  stamped, you can buy
train tickets to go back there any time

How about sidekicks? They must be able to come with you

\subsubsection{{\bfseries\sffamily TODO} better name than “Passport”}
\label{sec-1-21-1}
\subsection{Compass; Metal Detector; Dowsing Rod\hfill{}\textsc{COMPASS}}
\label{sec-1-22}

Like in Zelda, give or take. Some kind of Compass, Metal Detector,
and other item(s) that can be used to find certain types of prizes in
the world. Dowsing Rod.

\subsubsection{{\bfseries\sffamily TODO} develop the list of prizes and ways to search\hfill{}\textsc{COLLECTIONS:COMPASS}}
\label{sec-1-22-1}

Not every collectible item needs to be detectable in the same way
\section{The Story}
\label{sec-2}

“Why is it so dark?”

“In the beginning, it is always dark. … This one grain of sand; it is
all that is left of my vast empire.”

\subsection{The Back Story}
\label{sec-2-1}

A bunch of big tortoises found themselves on top of a magical
elephant graveyard, which turned out to be a gateway to a special
alternate dimension

BLAH BLAH

Somehow they mutated. Somehow, in a way that does not involve rats,
little alien guys with no legs, or bank drive-through tubes full
of jelly.

Seriously, I want to be \textbf{very careful} to avoid any potential for
comparison with the Teenage Mutant Ninja Turtles.

BLAH BLAH

So now, tortoises can cross through this one-way portal and find
themselves in TODO WORLD-NAME

Their birdy friends --- finches --- come along for the ride and serve
as a mail service.

A few other random critters might exist, but they're special-case
one-off beasties with very specific jobs.

The world is somehow TODO cloaked in some kind of impenetrable fog or
something so that we have a natural barrier that gradually disappears
as we (designers) add more terrain to the geography.

\subsubsection{{\bfseries\sffamily TODO} world name}
\label{sec-2-1-1}

\subsubsection{{\bfseries\sffamily TODO} why can't you go everywhere?}
\label{sec-2-1-2}

fog of war

magic barrier

edge of scroll radius doesn't really work in 3D
\subsection{Characters}
\label{sec-2-2}

There are a team of characters who provide the basic story-movement
capabilities for the game designers. They'll have the most developed
personalities and are regarded as being particularly “special” by all
of the other AI's.

They're also the original group that arrived first.

These basic characters also serve as models that players can
(hopefully) identify-with; and, by virtue of building their
personality types, other characters will in turn be able to exhibit
features of them.

They're the archetypes, from which all AI's descend.

They're intentionally based upon the models of how (we wish) players
will interact with the world, thus by merely existing and
participating, themselves, we'll see a continuous test of the game
world's ability to withstand players.

The current working models here have been drawn from the elemental
system of Chinese astrology as well as various psychological
documentation with a bit more scientific rigour; the various
categories will likely be merged into fewer characters as we
develop them.

\url{http://www.gamasutra.com/view/feature/6474/personality_and_play_styles_a_.php}

\url{http://www.yukaichou.com/gamification-study/user-types-gamified-systems/}

\subsubsection{{\bfseries\sffamily TODO} leader}
\label{sec-2-2-1}

A Leader.

The natural leader, capable of making a binding decision for
everyone; a monarch, a benevolent-dictator-for-life, and the leader of
the group; s/he takes everyone's opinions into account, but they
accept his/her decisions as law.
\subsubsection{{\bfseries\sffamily TODO} wacky inventor}
\label{sec-2-2-2}

\subsubsection{{\bfseries\sffamily TODO} earth person}
\label{sec-2-2-3}



Most Earth-type people are trustful, steady, loyal and
responsible. They are honest, religious, reliable, and keep
their promises.

The characteristics of Earth are standing still, being slow and
steady, and stationary. Therefore the Earth people don't like to move
their body and change their mind too often. They are honest,
trustworthy, and responsible. They also have good faith. They can sit
in one position for a long time. They accept religion since they like
rumination and meditation. They are slow to react. They like to
collect things, and enjoy their collectables at home, but
not outdoors.

People lacking Earth are selfish, insincere, and self-indulgent. They
tend to ignore other people's opinions. They might take immediate
advantage of others unethically. They don't care about the people
around them and do not keep their promises.

People with excess Earth are stubborn, inflexible, cheap, lonely,
plain and simple. They cannot keep their word either. 
\subsubsection{{\bfseries\sffamily TODO} metal person}
\label{sec-2-2-4}



Metal-type people are righteous, faithful, gallant and
chivalrous. They like to know and help people.

Metal is the substance in which internal particles squeeze and
condense together. There is a force from the outside to the inside
that keeps Metal hard. Metal reflects light, so it is shiny. It may
have a clear sound when it's hit. Therefore, a Metal-type person has
great strength, discipline and enough courage to aid needy people,
which may make them famous.

Metal people are brave, disciplined, trained, organized,
authoritative, determinative, routine, firm, resolute and have an
urge to win. They have a large sense of honor. They like to spread
their fame. They are sensitive in grief and sadness.

People that lack Metal are quiet, cautious and nervous. They tend to
think too much, and cannot make quick decisions. They do not express
what is on their minds in public, but will complain about
something afterwards.

People with excess Metal are often unkind and destructive. They are
brave but may have no resourcefulness and no intelligence. They have
little grace and no mercy. They do not keep secrets.
\subsubsection{{\bfseries\sffamily TODO} water person}
\label{sec-2-2-5}

Water-type people are smart, wise, frank, and resourceful. They have
good memories and think before they leap.

There are two types of Water: Floating Water and Still
Water. Floating Water makes people active and restless and feel
like traveling. Still Water makes people clam and peaceful.

People lacking Water are unstable, cowardly, narrow-minded, and have
no stamina. They lack intellect, good sense, understanding,
and foresight. They also tend to keep changing their mind.

People with excess Water are smart, sly, tricky and plot
dark schemes. They like to move or travel around and have a
sensual life. They are likely to dream too much, and keep changing
their mind as well. They also are only interested in what concerns
them and have no interest in outside world affairs.
\subsubsection{{\bfseries\sffamily TODO} wood person}
\label{sec-2-2-6}

Wood-type people are kind, steady, sympathy-sharing, understanding
and gentle. They like to help people and make donations.

The essence of Wood is a tree. The force inside the tree is like
growing upward to the sky. Trees often compete with each other
aggressively when growing up. Because of the characteristics of
trees, Wood-type people are steadfast, organized, logical, practical,
innovative, unique, assertive, fortitudinous, independent,
challenging and direct. They like to plan things, take action, and go
on adventures and challenges. They are perfectionists who tend to
push themselves to the limit and seek out the best, and take first
place in competition.

When their ambition and optimism is overwhelming, Wood-types can
become upset, nervous, unstable, impatient, intolerant, and lose
their humor by the frustration of many obstacles.

People lacking Wood essentially are weak in their opinions. They lack
the determination to change their options when situations
change. They are likely to be jealous when they don't have a strong
viewpoint of their own.

People with excess Wood are often inflexible, prejudiced and
biased. They tend not to accept or absorb others' opinions.
\subsubsection{{\bfseries\sffamily TODO} fire person}
\label{sec-2-2-7}



A Fire-type person is courteous, eloquent, polite and
expressive. They are good at compliments in speech.

Fire people are energetic, artistic, passionate, easily excitable,
and have a tendency to rapidly change emotional states as well as
become aggressive. They love sensation, drama and sentiment. They
seek joy, gratification and the attention of others. They like to be
invited to a party and dislike being alone.

People lacking Fire are aggressive but have no persistence. They are
inclined to have little confidence and have plenty of worries.

People with excess Fire are often talkative, overstated, overexcited,
overheated, sweet-talking, smart and restless. They tend to lecture
and offend others because of their short temper. 

\subsubsection{{\bfseries\sffamily TODO} killer person}
\label{sec-2-2-8}


Killers like to provoke and cause drama and/or impose them over other
players in the scope provided by the virtual world. Trolls, hackers,
cheaters, and attention farmers belong in this category, along with
the most ferocious and skillful PvP (player versus player) opponents.

Artisan/Killer: Finally, there are the Killers (or, as I prefer to
call them, Manipulators). These can be difficult to understand in a
gameplay context because most virtual worlds have encoded rules that
marginalize their play style as "griefing" (i.e., upsetting other
players) and try to prevent it. As Bartle puts it, "Killers get their
kicks from imposing themselves on others." He also points out that
Killers "wish only to demonstrate their superiority over
fellow humans."

This desire for power over everything in their world is most closely
echoed in the Keirseian description of Artisans, who (as their
temperament name suggests) delight in the skillfully artistic
manipulation of their environment. The Artisan/Killers are the
tool-users, the adrenaline junkies, the natural politicians, the
combat pilots, the high-stakes gamblers, and the negotiators
par excellence. They instinctively find and exploit advantages in any
tactical situation, and they express this need for dominance of their
world in order to retain the greatest amount of personal freedom
possible (External Change).

I believe a very good example of this can be found in Ryan
Creighton's "social engineering" of the coin-collecting game at the
Social Game Developers Rant of the 2011 Game Developers Conference. A
Guardian/Achiever would have played by the rules and raced around the
room begging others for their coins to try to win the game; an
Idealist/Socializer would have asked for coins as a way to meet new
people or help others win; and a Rational/Explorer would have sat
quietly watching the flow of coin exchanges to try to understand the
nature of the game. But an Artisan/Killer would instantly see how to
short-circuit the designed system, and, as a born negotiator, would
find it easy to persuade the person holding one of the bags of coins
to hand the whole thing over\ldots{} which is exactly what happened.

If the attendees needed to hear a rant from anyone, it would be the
Manipulator who is out there, just waiting to exploit any opportunity
to bring a little chaos to the carefully designed order of a
social game. (See Ryan's description of the event for a wonderful
first-hand account of gameplay from what appears to me to be a
classic Artisan/Killer perspective.)

A final note on the Keirsey/Bartle linkage: the Keirsey temperaments
and Bartle Types may appear not to line up directly where attitudes
toward other people are concerned. This is because the Bartle Types
were developed within a multi-player environment, which selects for
more extroverted, sociable gamers, while the temperaments include
both extroverts and introverts.

So, for example, the "Socializer" term that makes sense within the
Bartle Types for its emphasis on interacting with other people can
seem not to apply to an introverted Idealist who prefers to play
single-player games. These less-social Socializers are more likely to
prefer individualized entertainment or abstract games, making it
difficult to distinguish them from Rational/Explorer gamers. Closer
study is usually required to see whether their primary reason for
playing is to feel good (an Idealist preference) or to exercise their
thinking skills (a Rational goal).
\subsubsection{{\bfseries\sffamily TODO} achiever person}
\label{sec-2-2-9}

Achievers are competitive and enjoy beating difficult challenges
whether they are set by the game or by themselves. The more
challenging the goal, the most rewarded they tend to feel.

Guardian/Achiever: For the Guardian, the world is an insecure place,
so it's necessary to protect oneself by accumulating material
possessions\ldots{} just in case. Thus, Guardians focus on earning money,
on competing with others for resources perceived as scarce, on buying
nice things and maintaining them, on forming stable and hierarchical
group relationships, and generally on working hard to make their
place in the world secure by locking down their connections to the
world as possessions (External Structure).

Compare that to Bartle's description of Achievers: "Achievers regard
points-gathering and rising in levels as their main goal" and
"Achievers are proud of their formal status in the game's built-in
level hierarchy, and of how short a time they took to reach it."
Leveling up, leaderboards, and the accumulation of vast quantities of
looted items are all behaviors that are driven more by a
security-seeking motivation than by other motivations such as
powerful sensations, understanding or self-growth.

This explains why the Guardian/Achiever is willing to persist in long
stretches of "grind" that other kinds of gamers don't perceive as fun
at all. To this gamer, rewards should be proportional to the amount of
effort invested. When a game is designed around simple, well-defined
tasks that enable the competitive accumulation of status tokens, that
game is virtually guaranteed to attract security-seeking
Guardian/Achievers.
\subsubsection{{\bfseries\sffamily TODO} explorer person}
\label{sec-2-2-10}

Explorers like to explore the world – not just its geography but also
the finer details of the game mechanics. These players may end up
knowing how the game works and behave better than the game
creators themselves. They know all the mechanics, short-cuts, tricks,
and glitches that there are to know in the game and thrive on
discovering more.

Rational/Explorer: Rationals play in the same way that they do
everything else -- they find pleasure in discovering the organized
structural patterns behind raw data (Internal Structure). These can
be patterns in space (as in geography) or patterns in time (as in
morphology). Or they can be cause-and-effect patterns (entailment) or
relationship patterns (connections). Ultimately, it's all about
achieving a strategic understanding of the system as a whole thing.

As Bartle describes Explorers: "The real fun comes only from
discovery, and making the most complete set of maps in existence." Of
the core motivations -- sensation-seeking, security-seeking,
knowledge-seeking, and identity-seeking -- exploration as "discovery"
is most closely aligned with the Rational's
knowledge-seeking preference. For the Rational/Explorer, once the
principle behind the data is revealed, that's enough -- understanding
is its own reward. These gamers can enjoy imparting knowledge to
others, but no extrinsic reward for doing so is needed or expected.
\subsubsection{{\bfseries\sffamily TODO} socializer person}
\label{sec-2-2-11}

Socializers are often more interested in having relations with the
other players than playing the game itself. They help to spread
knowledge and a human feel, and are often involved in the community
aspect of the game (by means of managing guilds or role-playing, for
instance).

Idealist/Socializer: Socializers are described by Bartle as
"\ldots{} interested in people, and what they have to
say. \ldots{} Inter-player relationships are important \ldots{} seeing [people]
grow as individuals, maturing over time. \ldots{} The only ultimately
fulfilling thing is \ldots{} getting to know people, to understand them,
and to form beautiful, lasting relationships."

This is closely related to the Keirseian description of Idealists,
who are very aware of other people as part of their lifelong journey
of self-discovery (Internal Change). In a way, the highly imaginative
Idealists are always roleplaying; they are constantly creating images
of themselves (or others) that they feel they should model through
their own actions in order to produce the emotions in themselves that
they want to feel.
\subsubsection{{\bfseries\sffamily TODO} conqueror person}
\label{sec-2-2-12}

“I’ll beat any challenge”

Play: Hard Agon
Emotions: Anger/Fiero, (Fear?)
Skills: Strategic, Tactical \& Logistical 

The fiero-seeking Conqueror is the economic mainstay of the upper
market of videogames, thriving on a diet rich in First
Person Shooters. Challenge is the draw for this player – when the
complaint “it was too easy” is heard, it is heard from a
Conqueror. Fiero, the emotion of “triumph over adversity” requires
that the player be put through the ringer, pushed to their limits,
and as a result anger and (possibly) fear are likely to be
related emotions. It is likely that Conquerors are younger on average
than other players.
\subsubsection{{\bfseries\sffamily TODO} manager person}
\label{sec-2-2-13}

“I have to know how it works”

Play: Complex Ludus, Agon
Emotions: Contentment, Fiero
Skills: Strategic 

The strategic-minded manager is a complexity-seeking player. Games
with many rules, including both strategy games, and certain cRPGs,
are the mainstay of such a player, although adventure games will also
be enjoyed by many. Although fiero is likely to be a theme, the
Manager is less dependent upon this one emotion, and seeks the
satisfaction of knowledge or mastery, expressed through the feeling
of contentment. They can rack up serious hours on the games they
really love.


\subsubsection{{\bfseries\sffamily TODO} wanderer person}
\label{sec-2-2-14}


“Escape to another world”

Play: Mimicry, Paidia
Emotions: Wonder, Curiosity, (Fear?)
Skills: Tactical \& Diplomatic?

The escapist Wanderer seeks immersion in the sense of engagement with
an imaginary world. Such a player enjoys the beauty of fantasy
worlds, and is driven by a curiosity to see what is out there. Story
(specifically characters) is a greater drive than challenge, and
indeed the desire to know how the story ends may drive engagement
with any game. Fear may be enjoyed for the experience, in the manner
of a fairground spook house.
\subsubsection{{\bfseries\sffamily TODO} participant person}
\label{sec-2-2-15}


“Let’s play together”

Play: Agon? Paidia?
Emotions: Belonging, Amusement, Naches
Skills: Any?

The archetypal social player, the Participant doesn’t want to
play alone. Although competition (agon) is enjoyed, it is enjoyed
principally for the opportunity to be part of something taking place
between people. The need to belong, to be part of something, is
likely to be expressed most strongly with such a player.
\subsubsection{{\bfseries\sffamily TODO} hoarder person}
\label{sec-2-2-16}


“As much as I can get”

Play: Mimicry, Ludus?
Emotions: Greed, Contentment
Skills: Logistical

The logistically minded Hoarder cannot resist acquisition of
game resources. Likely found playing equipment-heavy cRPGs, as well
as MMORPGs, the Hoarder is a thorough player, gaining satisfaction
(and hence contentment) from the completion of “stamp collections”
and the like. When they finish a game, they usually find they have
accumulated an absurd amount of equipment, ammunition or money.
\subsubsection{{\bfseries\sffamily TODO} zoner person}
\label{sec-2-2-17}


“Time has lost all meaning”

Play: Simple Ludus, Alea, Ilinx?
Emotions: Excitement, Relief
Skills: Tactical 

Puzzle games are the zoner’s remit – lost in the flow of an abstract
game, they become intent upon the actions of the game they are
playing to the exclusion of all else. However, as much as they love
the games they play, they may not play for long period of
times. Short games played often is the nature of the experience.



\subsubsection{{\bfseries\sffamily TODO} juggernaut}
\label{sec-2-2-18}


“Knock ‘em down”

Play: Easy Agon, Mimicry, Paidia
Emotions: Amusement, Contentment, Excitement
Skills: Tactical?

The Juggernaut seeks a little resistance in the game they are
playing, but mostly wants to push through everything with comparative
(and amusing!) ease. A little excitement is desired, but the
Juggernaut isn’t looking for the degree of challenge that would
consistently supply fiero. Rather, they just want to play around –
often completely dominating the game they are playing. For the
Juggernaut, games aren’t about stress, they’re about unwinding. 
\subsubsection{{\bfseries\sffamily TODO} monster}
\label{sec-2-2-19}


“Evil is my middle name”

Play: Agon, Paidia
Emotions: Schadenfreude, Amusement
Skills: Strategic? Tactical 

The emotion of schadenfreude – taking delight in the misfortune of
others – drives the Monster. Mischief is their primary occupation –
“griefing” of strangers in a MMOG, and playful annoyance when
among friends. The Monster player is not interested in rules – except
in so much as they can find new ways to break them.
\subsubsection{{\bfseries\sffamily TODO} hotshot person}
\label{sec-2-2-20}


“The thrill of the ride”

Play: Ilinx, Mimicry
Emotions: Excitement, Relief, (Fear?)
Skills: Tactical

The master of high speeds and nail biting rides, the Hotshot is the
master of vertigo (ilinx). The ultimate payoff of victory (fiero)
will be enjoyed, but it is the experience of being at the brink of
control – the excitement (and perhaps fear) of being right on the
edge that is the driving force. 
\subsubsection{{\bfseries\sffamily TODO} agon person}
\label{sec-2-2-21}

\url{http://onlyagame.typepad.com/only_a_game/2006/03/the_challenge_o.html}
\subsubsection{{\bfseries\sffamily TODO} alea person}
\label{sec-2-2-22}

\url{http://onlyagame.typepad.com/only_a_game/2005/11/the_rituals_of_.html}
\subsubsection{{\bfseries\sffamily TODO} mimicry person}
\label{sec-2-2-23}

\url{http://onlyagame.typepad.com/only_a_game/2006/01/the_imagination.html}
\subsubsection{{\bfseries\sffamily TODO} ilinx person}
\label{sec-2-2-24}

\url{http://onlyagame.typepad.com/only_a_game/2006/05/the_joy_of_ilin.html}
\subsubsection{{\bfseries\sffamily TODO} ludus person}
\label{sec-2-2-25}

\url{http://onlyagame.typepad.com/only_a_game/2006/04/the_complexity_.html}
\subsubsection{{\bfseries\sffamily TODO} paida person}
\label{sec-2-2-26}

\url{http://onlyagame.typepad.com/only_a_game/2005/12/the_anarchy_of__1.html}

\subsection{Non-Tortoise Characters}
\label{sec-2-3}

\subsubsection{The Mockingbird}
\label{sec-2-3-1}

The Parrot is a “village elder” who gives advice to new players.
This is the tutorial section also.  New players come in throught the
portal and have to follow the mockingbird's instructions to
get through.

\begin{enumerate}
\item {\bfseries\sffamily TODO} name him
\label{sec-2-3-1-1}
\end{enumerate}
\subsubsection{Finches}
\label{sec-2-3-2}

The Finches travel around spreading whatever news we (admins) want
them to.  They're the Town Criers and also deliver
mail/packages.

\begin{enumerate}
\item {\bfseries\sffamily TODO} need a list of names
\label{sec-2-3-2-1}

I'm tempted to get themey and use, say, famous literary figures' names
\end{enumerate}
\subsubsection{{\bfseries\sffamily TODO} The Big Bad}
\label{sec-2-3-3}

The Big Bad wants  to do something cataclysmic to seek power for
himself, but is never seen ; 

He gives “gifts” to his minions

Probably the source of the thing that obstructs access to most of the map

Now he's a boogeyman, never seen; like Sauron, he lurks in the (TODO)
region and marshalls his forces without ever being seen, himself. All
of the characters fear the day that The Big Bad shows up again; they
don't know if he's regaining his evil magic as fast as they've been
building up their skills
\subsubsection{{\bfseries\sffamily TODO} the polluter}
\label{sec-2-3-4}

Goal  is to  pollute the world,  he leaves trash  around that
players need to wash up.

Very much Super Mario Sunshine. Sorry, Nintendo. 
\subsubsection{{\bfseries\sffamily TODO} the mail thief}
\label{sec-2-3-5}

tries to steal the mail and such, but if he gets caught, he'll drop
what he's taken. Otherwise, he'll take the packages and stash them in
various places throughout the world.

Players “catching” him will “free” up whatever packages he's stolen
but not yet hidden; the hidden things will be in trapped boxes.
Recovered, stolen goods should be dropped into any mailbox for
re-delivery.
\subsection{Currency\hfill{}\textsc{MONEY}}
\label{sec-2-4}

Need to come up with something to use for currency.

Scottish Pounds in the underwater area. Manatee money is only for
manatees. Teach them to deal with currency exchanges.

No money on the space station. Totally socialist.

When you "own" something like a house people know it's yours

if you drop other shit in public you might be littering or it might be
civic improvement —

\subsubsection{{\bfseries\sffamily TODO} how do the AI's decide?}
\label{sec-2-4-1}

if you drop other shit in public you might be littering or it might be
civic improvement —
\section{Minigames}
\label{sec-3}

Note, none of these are pop-up minigames or whatever. They're just
things to do in the game with the same interface as any other things.

\subsection{Football Core\hfill{}\textsc{FOOTBALL}}
\label{sec-3-1}

Basic  core that  we can  use different  rules variations  easily for:
soccer,  volleyball, baseball  or  softball  or kickball,  basketball,
possibly miniature golf or croquet, …
\subsection{House and Lot\hfill{}\textsc{HOUSE}}
\label{sec-3-2}

Lots: Miami-style. Each village has a  town square with a main Ave and
a main St that are analogous to 0 Ave / 0 St

Streets run E-W and have numbers running north or south.

Avenues run N-S and have numbers running east or west

Buy a lot and build a house on it

Build more rooms and decorate the yard
\subsection{The Mazes\hfill{}\textsc{MAZE}}
\label{sec-3-3}

Originally planned for Tootsville IV, but we never got decent art and
then the client-side support blew up. Yay Sam.

Mazes exist as:

\begin{itemize}
\item lilypads across the lake near the treehouses where the finches live

\item lava floes in the caves into the Lost World

\item under The Manor, Venetian theme; Venice, Roman catacombs, or the
Temple Of The Gods in Wind Waker.

\item The Manor maybe also a hedge maze

\item something like a maze around the Ranch?

\item going down hamster tubes on the space station

\item traversing a Coral Reef near the Manatee village \ldots{}
\end{itemize}


The Maze areas are made up of “tile”-like objects that are either
“positive” (places one \textbf{can} walk) or “negative” (one can only walk
where they are \textbf{not}); e.g. lilypads are a positive-space maze; hedge
mazes are negative-space; lava floes might be a combination of both,
with positive-space islands amid negative-space lava amid
positive-space.

Before a space transitions, there's an obvious transitional state;
e.g. lily pads partially submerged/arisen.

Some spaces may be “dangerous” and reset the player to something like
the start of the maze, but will have obvious “hints” to warn the
player of this.

Over time, the dynamic stepping stones are changed slightly: one
might appear or disappear randomly.

Stones won't disappear or become dangerous while someone is
standing upon them?

Keys and locked doors (of whatever kinds) may exist in each maze

\begin{verbatim}
The Locked Door can only be opened if the player possesses the
Key Pivitz and is standing nearby to it, within a
predefined area. Once unlocked, the Key is "destroyed" [but
therefore can be obtained again], and the door remains unlocked
[and passable] as long as the player remains nearby. Once the
player leaves the area [either by walking away or walking through
the door], the door locks again, requiring someone [same or other
player] find the Key to the door again.
\end{verbatim}

doors can also be teleports

\begin{verbatim}
The exits from each room re-enter other maze rooms more-or-less
at random: they are not laid out in a grid, and walking back
through the same door may not return you to the same room. Doors
may change where they go over time, perhaps daily or hourly.
\end{verbatim}

Possible alternate Key/Lock mechanism might be to obtain a Key and
require it to go to a Lock elsewhere in the maze, which then opens a
door distant from the lock for a given time (say, 5 minutes or
so).

NPC hinting, per the old Tootsville IV plan it went like

\begin{verbatim}
A Super Toot Bot will stand just inside the entrance from the
Castle with a short phrasebook explaining the basic rules, e.g.:
    The maze is changing all the time.
    If you can find the key, there's a prize behind the
    locked door.
    Watch out for slippery stones or you might be sent to
    Shaddow Falls!
    If you get lost, use your Compass to escape.
    Run away if you see The Big Bad! He'll steal your peanuts!

\end{verbatim}

Upsides: serves puzzle-solving players; fairly good replayability

Downsides: requires new assets [prizes, rooms] to be added from time
to time.
\subsection{Cleaning up after Smudge}
\label{sec-3-4}

Vacuum (?) up stains left by Smudge
\subsection{Tower Defense Picnic}
\label{sec-3-5}

One idea: columns of ants marching in. The "towers" throw
ant-foodstuffs at them: seeds and such. When the ants have "absorbed"
enough "hit points" [carbs?] worth of food, they turn around and
march peacefully away. "Sated" ants change colors and move faster;
they're not targeted by the weapons.

The player's "base," then, is a picnic basket, with a certain number
of peanuts in it. As the ants successfully take peanuts, they are
"sated" and rush off screen. Once the entire picnic has been stolen
by the ants, the game ends.

Since kids crave closure, we could have the game "end" after a
certain number of rounds.

This could be done now [Nightmare] as a minigame, or later [Osiris]
as an in-world game. 
\subsection{Digging Game}
\label{sec-3-6}

This was a knock-off of Astro-Pop.

Basically:

Move spaceship left or right across bottom of board.

Use "tractor beam" to pull down blocks which are the same
color. Collect blocks from one or more columns, then push them back
into another column by dropping them.

When enough blocks of the same colour collide, they "pop," and trigger
any adjacent special blocks to drop their prizes as well.

Note that the implementation of the pattern-search for the blocks
array was something that PopCap? did very badly in their first
implementation of this game, so we should watch out for "gotchas"
in that.

We could probably invert the screen vertically and have this as a
digging game. The blocks would be various chunks of earth; moving them
together causes little "cave ins." The prizes are carried up out of
the earth by grateful bugs [earthworms, ladybirds, etc.] to the player
walking across a scaffold at the top of the screen. The players could
them collect a certain type of prizes as a goal, once they receive
enough of these [gems?] they could move on to a higher level.
\subsection{Zuma/Luxor concept}
\label{sec-3-7}

Games like "Zuma," include the popular "Luxor" clone, we could easily
produce one of these.

The only difficulties we had in production of Zuma from PopCap? were:
asynchronous ball-rolling animations in the axis of forward travel
tended to cost a lot of CPU power; we went to a synchronous, general
animation for the seat-back systems; and the tunnels code gave
me trouble.  

Oh, wait, we're 3D now. Squelch that.
\subsection{Insaniquarium-inspired fish pond?}
\label{sec-3-8}

"Feed fish; fight aliens."

The bear kids were stoned or something, but it's an addictive
format. We should probably rework it into something other than an
aquarium, of course.

Basically, you drop fish-food into the tank, to feed the fish, who in
turn both grow, and crap out gold coins. Clicking coins collects them,
to give you money, to buy more fish [and some other helpful creatures
that assist in feeding and collecting coins].

From time to time, aliens beam into the tank, and you click on them
to LASER them until they explode back to their alien universe.

Both aliens and larger fish eat the smaller fish.

The object, naturally, is to have your fish survive for as long
as possible. 
\subsection{Mail\hfill{}\textsc{MAIL}}
\label{sec-3-9}

Leave mail in mailboxes; finches distribute

Write messages, wrap parcels

monsters tries to steal them
\subsection{Luigi's Mansion?}
\label{sec-3-10}
\subsection{Platformer areas}
\label{sec-3-11}
\subsection{the spaceship game\hfill{}\textsc{SPACESHIPS}}
\label{sec-3-12}

used to be called Zap's spaceship battle or something

Let's make it into “asteroids”

When there aren't human players, some robots will have to jump in, to
keep the space station from being damaged by the never-ending barrage
of falling rocks and space junk
\subsection{Dangerous areas: The Big Bad}
\label{sec-3-13}

On a pseudorandom schedule, The Big Bad will appear in one of the maze
rooms and steal peanuts from all players who don't flee fast
enough, e.g.:

\begin{itemize}
\item The Big Bad appears
\item Tendrils weather sets in
\item "Hahaha! I'll steal \# peanuts from you all!"
\item Wait for \# seconds
\item steal peanuts [\#givenuts -\# @room]
\item The Big Bad vanishes
\item Repeat randomly every 2 hours or so
\end{itemize}

Consideration: Perhaps not The Big Bad himself. I don't want to introduce
the character too early and end up making him a mere nuisance.

Perhaps having Targ or Welduh or something do this kind of harassment
makes more sense.

Allowing some use of Wishes to stop them would be good
\section{Places\hfill{}\textsc{MAP}}
\label{sec-4}

\begin{enumerate}
\item {\bfseries\sffamily TODO} map
\label{sec-4-0-0-1}

\subsection{Downtown}
\label{sec-4-1}

Tortoise St and James Ave

\subsubsection{Main plaza}
\label{sec-4-1-1}

\begin{enumerate}
\item The shopping area\hfill{}\textsc{SHOP}
\label{sec-4-1-1-1}

From 2009, I wrote:

9.6 Shopping

\begin{itemize}
\item Shopping is  now an  in-world action.  If desired,  storekeepers can
even be conversant NPC:s.

\item Players can walk  through shops and view items on  shelves, and then
purchase them by  clicking on the items and choosing  to buy them or
not. When clicked upon, store items  will pop up a detail sub-window
similar to the previous “web catalog” views.

\item This gains  us the  additional psychological advantage  that players
will  literally see  “everything” for  sale, rather  than only  ever
seeing a few items at a time. This makes the shops look much bigger,
even though  from an artwork perspective  we only need to  draw [and
then replicate] a couple of “shelves” graphics.

\item None  of the  shopping  functions require  special  handling on  the
client side, since  the stores are just ordinary 3D  rooms with some
special logic  on the server  side to handle purchasing  items [and,
possibly, speaking with the clarks at the stores].
\end{itemize}

We didn't really live up to it at the time, but the examples of
actually trying to do so in Tootsville (like Capes \& Cowls) seemed to
validate the statement.
\item Restaurants
\label{sec-4-1-1-2}

a few of these.

\begin{enumerate}
\item {\bfseries\sffamily TODO} Cooking\hfill{}\textsc{MINIGAME}
\label{sec-4-1-1-2-1}

Cooking game, like Cooking Mama.

Baking  Mama. 
\item {\bfseries\sffamily TODO} Waiting Tables\hfill{}\textsc{MINIGAME}
\label{sec-4-1-1-2-2}
Waiting tables game, like Diner Dash
\item {\bfseries\sffamily TODO} Washing Dishes\hfill{}\textsc{MINIGAME}
\label{sec-4-1-1-2-3}
Washing dishes game
\end{enumerate}
\item Train station\hfill{}\textsc{TRAINS}
\label{sec-4-1-1-3}
main station
\item Theater
\label{sec-4-1-1-4}

The Theater …  will need some ideas here, but  Archive.org and <video>
might   be  a   cool   fit,  if   we   can  overcome   browser-related
issues. Running old cartoons and such.

Yes, this is the same idea I used for Tootsville when Louis's
original content didn't come through, so it's kinda “been there, done
that,” but it \textbf{is} an idea.

\begin{enumerate}
\item {\bfseries\sffamily TODO} Video "recording"?\hfill{}\textsc{MINIGAME}
\label{sec-4-1-1-4-1}

What if players could act out scenes on a soundstage and record them,
and then enter a contest of some kind to have their videos
posted here?

Kinda rehashes something the Tootsville players had "invented" on
their own.

\begin{enumerate}
\item {\bfseries\sffamily TODO} YouTube?
\label{sec-4-1-1-4-1-1}

It's even possible we could share out the "best of" to YouTube.
\end{enumerate}
\item {\bfseries\sffamily TODO} Video playback?
\label{sec-4-1-1-4-2}

Tech issues? <VIDEO> element support, transcoding, bandwidth
\end{enumerate}
\item Inventor's Shop
\label{sec-4-1-1-5}

crazy Rube Goldberg physics engine tests
\item Fairgrounds
\label{sec-4-1-1-6}
for events

\item Hall Of Heroes
\label{sec-4-1-1-7}

statues for famous players?
\item Spaceport
\label{sec-4-1-1-8}

the launch/landing pads for rockets to the space station are here

\begin{enumerate}
\item {\bfseries\sffamily TODO} Launch minigame\hfill{}\textsc{MINIGAME}
\label{sec-4-1-1-8-1}

minigame to help run the countdown from Mission Control; if you fail,
the rocket launch will abort and you'll have to wait for them to
refuel; nobody else can go to the space station until the next launch
window
\end{enumerate}
\end{enumerate}
\subsubsection{School?}
\label{sec-4-1-2}

need some actual contents here

walk the halls

get a locker

take a class…

honour roll?

\begin{enumerate}
\item {\bfseries\sffamily TODO} design classes\hfill{}\textsc{RESEARCH:MINIGAME}
\label{sec-4-1-2-0-1}

actual minigames based on learning things in classes

\begin{itemize}
\item maths: basic arithmetic, fractions

\item money handling (US currency, coins, arithmetic)

\item basic physics: simple tools and such (lever, plane, pulley)

\item English grammar and vocabulary (?)

\item Spanish vocabulary

\item penmanship? tough sell with a mouse but OK on touchscreens or so
forth, but will require deeper level coöperation with the front-end

\item drama class? Acting out simple plays? shadowcasting?

\item very simple real world history/geography. Where is Rome? Where was
the Roman Empire? Where was Alexander's Empire? Who were the
Ancient Egyptians?
\end{itemize}
\end{enumerate}
\subsubsection{Sports Arena}
\label{sec-4-1-3}

\begin{enumerate}
\item {\bfseries\sffamily TODO} Soccer game\hfill{}\textsc{FOOTBALL:MINIGAME}
\label{sec-4-1-3-0-1}

\item {\bfseries\sffamily TODO} Baseball game\hfill{}\textsc{FOOTBALL:MINIGAME}
\label{sec-4-1-3-0-2}

\item {\bfseries\sffamily TODO} Archery?\hfill{}\textsc{MINIGAME}
\label{sec-4-1-3-0-3}

\item {\bfseries\sffamily TODO} Track?\hfill{}\textsc{MINIGAME}
\label{sec-4-1-3-0-4}

(tough sell. speed stats?)
\item {\bfseries\sffamily TODO} Relay race?\hfill{}\textsc{MINIGAME}
\label{sec-4-1-3-0-5}

(maybe better than track)
\item {\bfseries\sffamily TODO} Mini Golf!\hfill{}\textsc{FOOTBALL:MINIGAME}
\label{sec-4-1-3-0-6}

Based on the football model; but knock the ball through various
obstacles 
\item {\bfseries\sffamily TODO} Kids' Calisthenics\hfill{}\textsc{RESEARCH:MINIGAME}
\label{sec-4-1-3-0-7}

teach them useful stuff. I dunno. VAGUE idea.

Tag or Kick the Can or so forth

interval training?
\end{enumerate}
\subsection{The Beach}
\label{sec-4-2}

Peninsula to the south, around the bay of TODO, where the Manatees live

Note: tortoises can swim.

\subsubsection{{\bfseries\sffamily TODO} Volleyball\hfill{}\textsc{MINIGAME}}
\label{sec-4-2-1}

Volleyball game: Character  keeps score, join a team and  play a game,
same as soccer and baseball basically
\subsubsection{{\bfseries\sffamily TODO} watercraft}
\label{sec-4-2-2}

maybe jet-skiing, small boat rental

\begin{enumerate}
\item {\bfseries\sffamily TODO} Pirate Ship\hfill{}\textsc{LATER}
\label{sec-4-2-2-1}

Pirate Ship: when it arrives (future), it'll go on tours of the area
to various small desert islands; the pirates will collect and bury
treasure, maybe give out treasure maps with X marks the spot
\end{enumerate}
\subsection{Manatee Village}
\label{sec-4-3}

Manatee residents?  because  Manatees are huggable

The currency here are different than on land, and only useful in the
sea, like how Scottish Pounds are (were?) useless in London.

\subsubsection{Underwater lab}
\label{sec-4-3-1}

Let's see about  getting some “wonders of the undersea  world” kind of
educational content in here, too

\subsubsection{{\bfseries\sffamily TODO} Coral Reef Maze\hfill{}\textsc{MINIGAME:MAZE}}
\label{sec-4-3-2}
\subsection{Space station}
\label{sec-4-4}

need a launchpad to get here

rockets launch and land periodically

\subsubsection{{\bfseries\sffamily TODO} asteroid mining\hfill{}\textsc{MINIGAME}}
\label{sec-4-4-1}

The space game should be  reproduced; Buy spaceships, fly them around,
play asteroids
\subsubsection{{\bfseries\sffamily TODO} hamster tube maze\hfill{}\textsc{MINIGAME:MAZE}}
\label{sec-4-4-2}

Hamster tubes through the station; like Jeffries tubes, but for
tortoises. They get re-arranged by space station engineers all the
time, this is one of the ‘always changing maze’ areas.
\subsubsection{{\bfseries\sffamily TODO} contact SpaceX about allowing use of Dragon/Falcon in game}
\label{sec-4-4-3}

permission to use Dragon vehicles to access?

It'd be cool to include actual spacecraft, just need a trademark
license.  Maybe Musk is feeling excitable about merchandising and will
feel like manufacturing some spacecraft toys :-)
\subsection{jungle, savannah, prarie}
\label{sec-4-5}

jungle exploration — jungle turns into the forest — this is
yet another maze area that is always changing

Savannah turns into prarie for ranch

Digging in the savannah you might find different things buried
\subsection{Forest}
\label{sec-4-6}


Elders' Cottage for healing

Hopping around tree-tops, swinging on vines, platformer-style play

\subsubsection{Butterflies\hfill{}\textsc{MINIGAME}}
\label{sec-4-6-1}

gather wild butterflies — chase them around, with a net or a bubble
wand, catch with a net and they'll give you rewards, or you can do
something different with them

fairy in bottle?
\subsubsection{Finch Village}
\label{sec-4-6-2}

singing frogs?

maze of super-sized lily pads across the lake, always changing

Located near the big lake

\subsubsection{caves lead to Lost World\hfill{}\textsc{MINIGAME:MAZE}}
\label{sec-4-6-3}

avoid lava floes, which shift around from time to time (maze)

navigate them on floating rock islands
\subsubsection{Lost World}
\label{sec-4-6-4}

dinosaurs roaming around

maybe some in-world gaming things regarding the dinos?
\subsection{Ranch}
\label{sec-4-7}

Homes out here on the prarie
\subsection{The Manor}
\label{sec-4-8}

big old French Chateau kind of thing
\subsection{Snowy place}
\label{sec-4-9}

ice and snow and stuff
\section{Time, Weather}
\label{sec-5}

\subsection{Time}
\label{sec-5-1}

Time flows in an $18 \frac{2}{3}$-hour day. ($\frac{56}{3}$)

Perhaps we should invent a “clock” that treats $18 \frac{2}{3}$ hour
days normally?  Like, we stretch the hours such that 24:00 = midnight;
12:00 = noon; 06:00 and 18:00 are sunrise and sunset;

By that scale, an hour = $\frac{140}{3}$ minutes or $46 \frac{2}{3}$
minutes of real world time.

If  we  call  these  “star  hours”  we  can  even  vary  the  seasonal
sunrise/sunset tables based on some arbitrary real-world latitude

\subsubsection{Non$\backslash$=es}
\label{sec-5-1-1}

“Weeks” have 9 days (Non$\backslash$=es like Roman calendar)

Every 9th day is Market Day

The days need renaming based on the basic characters we end up having


The days are: (mapped to Eastern TZ)

\begin{enumerate}
\item 1 Star Day
\label{sec-5-1-1-1}

(Sat/Sun M - Sun 1846)

The sun is a star
\item 2 Sport Day
\label{sec-5-1-1-2}

(Sun 1846 - Mon 1334)

Aligns with football night
\item 3 Music Day
\label{sec-5-1-1-3}

(Mon 1333 - Tue 0820)

New music releases on Mondays

Special events: ?
\item 4 Electric Day
\label{sec-5-1-1-4}

(Tue 0820 - Wed 0306)


\item 5 Science Day
\label{sec-5-1-1-5}

(Wed 0306 - Wed 2153)

new comic book day

\item 6 Animal Day
\label{sec-5-1-1-6}

(Wed 2153 - Thu 1639)


\item 7 Friends Day
\label{sec-5-1-1-7}

(Thu 1639 - Fri 1126)

new movie releases Thursday midnights

ok, maybe we'll release \textbf{really old} movies instead?
\item 8 Flowers Day
\label{sec-5-1-1-8}

(Fri 1126 - Sat 0613)

Special events: ?
\item 9 Market Day
\label{sec-5-1-1-9}

(Sat 0613 - Sat/Sun M)

the main day for major special events

The Market  Day should align  with Saturday for  the UK through  Au as
much as possible, so when we  do special things for Market Day they're
accessible to the most players
\end{enumerate}
\subsection{Weather}
\label{sec-5-2}

Weather flows pseudo-randomly. Maybe we just pick random cities in the
real  world and  follow its  weather. 

Weather conditions: rain; cloudiness; fog; snow
\subsection{Other minigame ideas?}
\label{sec-5-3}

\subsubsection{Pac-Man.}
\label{sec-5-3-1}

If we're going to do Pac-Man, do it like Katamari Damacy. That was a
cool Pac-Man clone; and it can be done in-world.

Perhaps a “harvesting” theme.

Problem is resetting the “board”
\subsubsection{Tic Tac Toe}
\label{sec-5-3-2}

do I have to explain this one?

Actually, we could give users the ability to play Tic-Tac-Toe
multi-player by positioning oversized tokens in the game world.

Kinda like Hobo Chess.
\url{http://i14.photobucket.com/albums/a349/benjistour/P1010228.jpg}

\subsection{Building the code}
\label{sec-5-4}

\subsubsection{Directory Structure}
\label{sec-5-4-1}

\begin{enumerate}
\item \texttt{build}
\label{sec-5-4-1-1}

Everything that gets temporarily constructed during a \texttt{make} should
be in here, and is wiped out by \texttt{make clean}
\item \texttt{dist}
\label{sec-5-4-1-2}

The actual final distribution files go in here.

There are two bundles, one for code, one for pure assets. Eventually
these would be on the game server and on a static asset
server/network/farm potentially.

They're also versioned separately, see the control-cards § at top of
Makefile
\item \texttt{doc}
\label{sec-5-4-1-3}

documentation sources live here

\begin{enumerate}
\item \texttt{doc/design}
\label{sec-5-4-1-3-1}

docs for game designers

how to use editing functions and such
\item \texttt{doc/devel}
\label{sec-5-4-1-3-2}

docs for people working on the game code here. like this mess you're
reading, it lives there.

\begin{enumerate}
\item {\bfseries\sffamily TODO} devel docs into \texttt{Makefile}
\label{sec-5-4-1-3-2-1}

that includes this big mess you're reading now
\end{enumerate}

\item \texttt{doc/legal}
\label{sec-5-4-1-3-3}

contractual stuff and all of that

\begin{enumerate}
\item {\bfseries\sffamily TODO} \texttt{doc/legal/licenses}
\label{sec-5-4-1-3-3-1}

Licenses for all program components and assets need to make their way
here with some kind of system for concatenating them to generate the
“full spiel”
\end{enumerate}
\item \texttt{doc/user}
\label{sec-5-4-1-3-4}

user-facing docs

\begin{itemize}
\item marketing
\item help files
\end{itemize}

\begin{enumerate}
\item {\bfseries\sffamily TODO} user docs into \texttt{Makefile}
\label{sec-5-4-1-3-4-1}

they need to be distributed with the game assets

\texttt{ui-help.lisp} is going to reference them as well so the names need
to be stable
\item {\bfseries\sffamily TODO} user docs templating system
\label{sec-5-4-1-3-4-2}

probably want to write them in \LaTeX{} or something and render via a
templating system to give consistent and manageable headers, footers,
CSS, whatever
\end{enumerate}
\end{enumerate}
\item \texttt{src}
\label{sec-5-4-1-4}

All source code should pretty much live here

anything that's getting compiled (or minified), for sure

\begin{enumerate}
\item \texttt{src/css}
\label{sec-5-4-1-4-1}

CSS to be minified

\begin{enumerate}
\item {\bfseries\sffamily TODO} consider Closure CSS or LESS or something?
\label{sec-5-4-1-4-1-1}
\end{enumerate}
\item \texttt{src/ps}
\label{sec-5-4-1-4-2}

Parenscript code: the entire front-end lives here.

Keep it flat, for now, the compiler doesn't descend subdirectories

cross-module dependencies are not resolved in any way; don't share
variables and things if you can help it


\begin{enumerate}
\item \texttt{src/ps/00-preamble.lisp}
\label{sec-5-4-1-4-2-1}

The macros needed by other Parenscript files
\item \texttt{src/ps/01-preamble.js}
\label{sec-5-4-1-4-2-2}

Common JS code that hasn't been converted to PS yet

gets glued to the front of the concatenated js files before
minification
\item \texttt{src/ps/asset-loader.lisp}
\label{sec-5-4-1-4-2-3}

pre-load assets of all kinds as needed

\begin{enumerate}
\item {\bfseries\sffamily TODO} show loading indicator
\label{sec-5-4-1-4-2-3-1}

partially done

needs some TLC but it's a viable start

see how it actually works when we start loading shit
\end{enumerate}
\item \texttt{src/ps/gl-utils.lisp}
\label{sec-5-4-1-4-2-4}


all the WebGL-related stuff goes in here

might end up with the 3D sound stuff as well

mostly should be just some glue and helper functions in the end

\begin{enumerate}
\item {\bfseries\sffamily TODO} camera movement
\label{sec-5-4-1-4-2-4-1}

camera angle relatively doesn't change, but has to scroll around to
follow the player

also change the zoom level to bring up panoramas or pull in closer
\item {\bfseries\sffamily TODO} translucency
\label{sec-5-4-1-4-2-4-2}

when player moves behind things, make them translucent
\item {\bfseries\sffamily TODO} indoors rooms
\label{sec-5-4-1-4-2-4-3}

\item {\bfseries\sffamily TODO} billboarded vectors
\label{sec-5-4-1-4-2-4-4}

might not need these so much after all
\item {\bfseries\sffamily TODO} billboarded text labels
\label{sec-5-4-1-4-2-4-5}

mostly for character names and maybe other selected items

try to keep them from getting obstructed, and scale based on browser
text size perhaps?
\item {\bfseries\sffamily TODO} speech bubbles
\label{sec-5-4-1-4-2-4-6}

basically a special case of billboarded text / billboarded vectors,
but an important special case

when they fade out, it would be nice to show some kind of affordance
that they're going away to the chat log area
\item {\bfseries\sffamily TODO} entity picker
\label{sec-5-4-1-4-2-4-7}

needed for various other things, perhaps allow hover highlighting and
display name if it has one (and isn't already displayed)
\end{enumerate}
\item \texttt{src/ps/network.lisp}
\label{sec-5-4-1-4-2-5}

\begin{enumerate}
\item {\bfseries\sffamily TODO} connect to server WebSockets
\label{sec-5-4-1-4-2-5-1}

\item {\bfseries\sffamily TODO} exchange login credentials
\label{sec-5-4-1-4-2-5-2}

\begin{enumerate}
\item {\bfseries\sffamily TODO} log in as sidekick
\label{sec-5-4-1-4-2-5-2-1}

\item {\bfseries\sffamily TODO} log in using openid?
\label{sec-5-4-1-4-2-5-2-2}

\item {\bfseries\sffamily TODO} log in using password
\label{sec-5-4-1-4-2-5-2-3}
\end{enumerate}

\item {\bfseries\sffamily TODO} login box
\label{sec-5-4-1-4-2-5-3}

needs to handle all in one easy place:

\begin{itemize}
\item MOTD
\item QoS indicators
\item log in with user name + password
\item log in with openid
\item log in as a sidekick
\item register new user
\item recover forgotten user name / reset password
\item general link to m+b subsystem
\end{itemize}
\item {\bfseries\sffamily TODO} invite sidekick
\label{sec-5-4-1-4-2-5-4}

\begin{enumerate}
\item {\bfseries\sffamily TODO} sidekick share via social media?
\label{sec-5-4-1-4-2-5-4-1}

\item {\bfseries\sffamily TODO} sidekick share via other channels?
\label{sec-5-4-1-4-2-5-4-2}
\end{enumerate}
\item {\bfseries\sffamily TODO} redirect to M\&B system for registration
\label{sec-5-4-1-4-2-5-5}

\item {\bfseries\sffamily TODO} redirect to M\&B system for forgotten passwords
\label{sec-5-4-1-4-2-5-6}

\item {\bfseries\sffamily TODO} accept entity data and populate simulation
\label{sec-5-4-1-4-2-5-7}

\item {\bfseries\sffamily TODO} send actions to the server
\label{sec-5-4-1-4-2-5-8}

\item {\bfseries\sffamily TODO} check for disconnections
\label{sec-5-4-1-4-2-5-9}

\item {\bfseries\sffamily TODO} auto-reconnect
\label{sec-5-4-1-4-2-5-10}

\item {\bfseries\sffamily TODO} diagnostics for disconnections
\label{sec-5-4-1-4-2-5-11}

\item {\bfseries\sffamily TODO} integrate with offline-detection
\label{sec-5-4-1-4-2-5-12}

\item {\bfseries\sffamily TODO} polyfill for browsers without WebSockets ?
\label{sec-5-4-1-4-2-5-13}


\item {\bfseries\sffamily TODO} migration support
\label{sec-5-4-1-4-2-5-14}

gentle migration:

\begin{itemize}
\item First, connect to the new node
\item Test that connection
\item Transfer it to being the primary
\item Then, close the old one
\end{itemize}
\end{enumerate}
\item {\bfseries\sffamily TODO} \texttt{src/ps/mb.lisp}
\label{sec-5-4-1-4-2-6}

\begin{enumerate}
\item {\bfseries\sffamily TODO} registration
\label{sec-5-4-1-4-2-6-1}

\item {\bfseries\sffamily TODO} change gecos info
\label{sec-5-4-1-4-2-6-2}

\item {\bfseries\sffamily TODO} change password
\label{sec-5-4-1-4-2-6-3}

\item {\bfseries\sffamily TODO} change openid binding
\label{sec-5-4-1-4-2-6-4}

\item {\bfseries\sffamily TODO} view account activity
\label{sec-5-4-1-4-2-6-5}

\item {\bfseries\sffamily TODO} parental controls
\label{sec-5-4-1-4-2-6-6}

\begin{enumerate}
\item {\bfseries\sffamily TODO} create child account
\label{sec-5-4-1-4-2-6-6-1}

\item {\bfseries\sffamily TODO} annex child account
\label{sec-5-4-1-4-2-6-6-2}

\item {\bfseries\sffamily TODO} edit child account gecos
\label{sec-5-4-1-4-2-6-6-3}

\item {\bfseries\sffamily TODO} set restrictions on child account
\label{sec-5-4-1-4-2-6-6-4}

\begin{enumerate}
\item {\bfseries\sffamily TODO} time limits
\label{sec-5-4-1-4-2-6-6-4-1}

\item {\bfseries\sffamily TODO} sidekick limiter
\label{sec-5-4-1-4-2-6-6-4-2}

\item {\bfseries\sffamily TODO} sidekick required / baby-sat mode
\label{sec-5-4-1-4-2-6-6-4-3}
\end{enumerate}
\item {\bfseries\sffamily TODO} assign other guardians to child
\label{sec-5-4-1-4-2-6-6-5}

\item {\bfseries\sffamily TODO} view child account activity
\label{sec-5-4-1-4-2-6-6-6}


\item {\bfseries\sffamily TODO} disable child account
\label{sec-5-4-1-4-2-6-6-7}

\item {\bfseries\sffamily TODO} delete child account
\label{sec-5-4-1-4-2-6-6-8}
\end{enumerate}
\item {\bfseries\sffamily TODO} delete account
\label{sec-5-4-1-4-2-6-7}
\end{enumerate}
\item \texttt{src/ps/string-utils.lisp}
\label{sec-5-4-1-4-2-7}

The big thing here is for the string table accessed via \texttt{(msg)} in
the code.

All user-visible strings should be in that message catalog, and we'll
later be able to translate it en masse

We use a simple formatting thing for these that is similar to Java's
String.format as well, so re-ordering words in the message won't
break on params

\begin{enumerate}
\item {\bfseries\sffamily TODO} analogue of \textasciitilde{}R
\label{sec-5-4-1-4-2-7-1}

Actually I think I like the idea of having a few variants

\begin{itemize}
\item spell out words or fall back on numerals when they're “too long,”
as defined by the language. In English, we usually stop at
“nineteen” … “20” boundary.
\item abbreviate to no more than 2 significant figures, with
rounding. “two thousand” for 2,000; “about two thousand” for
1,994
\item the proper pluralization system described below
\end{itemize}
\item {\bfseries\sffamily TODO} handle pluralizations in a sane way
\label{sec-5-4-1-4-2-7-2}

test cases:

(and (equal "one car" (plural 1 'car))
     (equal "two cars" (plural 2 'car))
     (equal "un auto" (with-language :es (plural 1 'car)))
     (equal "один машина" (with-language :ru (plural 1 'car)))
     (equal "nine cars" (plural 9 'car))
     (equal "nineteen cars" (plural 19 'car))
     (equal "20 cars" (plural 20 'car))
     (equal "no cars" (plural 0 'car))
     (equal "four children" (plural 4 'child))
     (equal "four red houses" (plural 4 '(house red))))

Note that Russian counting is about the worst-case-scenario because
it's not going to fit the singular/plural pattern like EN, ES, FR.

Note also that Asian counting requires that the symbol → word
dictionary provide the counting words type.
\item {\bfseries\sffamily TODO} handle full inflections, including plurals
\label{sec-5-4-1-4-2-7-3}

needed for some languages like DE, RU, GR, LA

not usually needed for EN, ES, FR, JA, CN 

\textbf{EXCEPT} in EN, ES, FR needed for pronouns only (I/me, he/him \&c in
EN; tu/te/ti in ES…)
\end{enumerate}
\item {\bfseries\sffamily TODO} \texttt{src/ps/ui-help}
\label{sec-5-4-1-4-2-8}

user interface helper stuff

dialog box creation and all

should be doing this using nice HTML overlays within the page

possibly also a way to load help screens without opening new
windows/tabs

\begin{enumerate}
\item {\bfseries\sffamily TODO} pop-up message
\label{sec-5-4-1-4-2-8-1}

\item {\bfseries\sffamily TODO} multiple choice buttons yes/no ok/cancel dialog
\label{sec-5-4-1-4-2-8-2}

\item {\bfseries\sffamily TODO} help viewer
\label{sec-5-4-1-4-2-8-3}

\item {\bfseries\sffamily TODO} launch external page gently
\label{sec-5-4-1-4-2-8-4}

evading pop-up blockers and the like
\end{enumerate}
\item {\bfseries\sffamily TODO} \texttt{src/ps/controls}
\label{sec-5-4-1-4-2-9}

UI controls

tap-n-talk

\begin{enumerate}
\item {\bfseries\sffamily TODO} selected inventory item display
\label{sec-5-4-1-4-2-9-1}

\begin{enumerate}
\item {\bfseries\sffamily TODO} engage paperdoll display
\label{sec-5-4-1-4-2-9-1-1}

\item {\bfseries\sffamily TODO} click-to-use item, no target
\label{sec-5-4-1-4-2-9-1-2}

\item {\bfseries\sffamily TODO} directional item
\label{sec-5-4-1-4-2-9-1-3}

\item {\bfseries\sffamily TODO} target-positional item
\label{sec-5-4-1-4-2-9-1-4}

\item {\bfseries\sffamily TODO} target-entity item
\label{sec-5-4-1-4-2-9-1-5}

\item {\bfseries\sffamily TODO} discrete counter
\label{sec-5-4-1-4-2-9-1-6}

\item {\bfseries\sffamily TODO} scalar display
\label{sec-5-4-1-4-2-9-1-7}
\end{enumerate}
\item {\bfseries\sffamily TODO} tap-n-talk
\label{sec-5-4-1-4-2-9-2}

\begin{enumerate}
\item {\bfseries\sffamily TODO} load possible vocabulary from server
\label{sec-5-4-1-4-2-9-2-1}

\item {\bfseries\sffamily TODO} allow user selections
\label{sec-5-4-1-4-2-9-2-2}

\item {\bfseries\sffamily TODO} compose sexp and send
\label{sec-5-4-1-4-2-9-2-3}

\item {\bfseries\sffamily TODO} chat log
\label{sec-5-4-1-4-2-9-2-4}
\end{enumerate}
\end{enumerate}
\item {\bfseries\sffamily TODO} \texttt{src/ps/paperdoll}
\label{sec-5-4-1-4-2-10}

inventory/paperdoll system
\end{enumerate}

\item \texttt{src/romans}
\label{sec-5-4-1-4-3}

The server components

The software components are named for (in)famous figures of Roman
history. Note, if you are familiar with the components in Romance 1.x,
some of these may have changed in their tasks slightly.

\begin{enumerate}
\item {\bfseries\sffamily TODO} need to nail down the MQ service type
\label{sec-5-4-1-4-3-1}

ZeroMQ and RabbitMQ seem to be my favourites for the moment

Requirements:

\begin{itemize}
\item FAST
\item FUCKING FAST
\item Easy to marshall arbitrary stream data OR BSON data
\item BSON is nicer but I don't much care
\item Flexible distribution grid
\item Lossless
\item In order of preference: Quicklisp sources that work; Lisp sources
that work; clear enough docs to write Lisp bindings without crying
\item Doesn't require PhD to install/configure
\end{itemize}
\item Appius
\label{sec-5-4-1-4-3-2}

Appius Claudius Caecus handles network I/O. All socket connections
from clients are routed through Appius, and into the message queues
for the game itself.

Appius Claudius Caecus was notable for building a major road out of
Rome, the Appian Way, as well as being blind, and twice Consul.

\begin{enumerate}
\item {\bfseries\sffamily TODO} socket-activation server
\label{sec-5-4-1-4-3-2-1}

working with SystemD for TCP streams (and WebSockets streams as well?)
\item {\bfseries\sffamily TODO} TCP stream binding
\label{sec-5-4-1-4-3-2-2}

\begin{itemize}
\item return packets the same way regardless of TCP or WS
\end{itemize}
\item {\bfseries\sffamily TODO} WebSockets binding
\label{sec-5-4-1-4-3-2-3}

how to listen? native listener seems best, is that Kosher with
same-origin policy?

if not, how do we tie in to the HTTP server enough to make
same-origin happy? serve the live HTML from the chat servers?

some experimentation will be required for this bit

\begin{itemize}
\item return packets the same way regardless of TCP or WS
\end{itemize}
\item {\bfseries\sffamily TODO} binding to MQ
\label{sec-5-4-1-4-3-2-4}

\item {\bfseries\sffamily TODO} BSON coding for packets
\label{sec-5-4-1-4-3-2-5}

\begin{enumerate}
\item {\bfseries\sffamily TODO} BSON-to-MQ
\label{sec-5-4-1-4-3-2-5-1}

\item {\bfseries\sffamily TODO} MQ-to-BSON
\label{sec-5-4-1-4-3-2-5-2}
\end{enumerate}
\item {\bfseries\sffamily TODO} JSON coding for packets
\label{sec-5-4-1-4-3-2-6}

\begin{enumerate}
\item {\bfseries\sffamily TODO} JSON-to-MQ
\label{sec-5-4-1-4-3-2-6-1}

\item {\bfseries\sffamily TODO} MQ-to-JSON
\label{sec-5-4-1-4-3-2-6-2}
\end{enumerate}
\item {\bfseries\sffamily TODO} QoS indicators
\label{sec-5-4-1-4-3-2-7}

\item {\bfseries\sffamily TODO} connection pool moderation
\label{sec-5-4-1-4-3-2-8}

when one guy's getting too many connections and another one is light,
alter the usual round-robin selection to balance the load
\item {\bfseries\sffamily TODO} migration support
\label{sec-5-4-1-4-3-2-9}

when bringing up/down Appius nodes, migrate users around to balance
the load
\end{enumerate}
\item Asinius
\label{sec-5-4-1-4-3-3}

Asinius handles connectivity to the Postgres database server, for
long-term storage and disaster recovery.

Gaius Asinius Pollio was a consul noted for constructing the first
public library in Rome, the Atrium Libertatis, as a posthumous
favor to Caesar.

\begin{enumerate}
\item {\bfseries\sffamily TODO} MQ-to-Postmodern
\label{sec-5-4-1-4-3-3-1}

\item {\bfseries\sffamily TODO} Postmodern-to-MQ
\label{sec-5-4-1-4-3-3-2}

\item {\bfseries\sffamily TODO} QoS indicators
\label{sec-5-4-1-4-3-3-3}

\item {\bfseries\sffamily TODO} QoS indicators extracted from Postgres
\label{sec-5-4-1-4-3-3-4}

\item {\bfseries\sffamily TODO} Postmodern schema mapping to entity data
\label{sec-5-4-1-4-3-3-5}
\end{enumerate}
\item Caesar
\label{sec-5-4-1-4-3-4}

Caesar oversees the system on which it is running, and ensures that
sufficient resources are available for uninterrupted
operations. Caesar may terminate workers when they are no longer
needed, or requisition additional resources (such as starting a new
virtual machine or requesting additional storage space)
when necessary.

Gaius Julius Caesar was known as a famous general.

\begin{enumerate}
\item {\bfseries\sffamily TODO} collation of QoS reports
\label{sec-5-4-1-4-3-4-1}

\item {\bfseries\sffamily TODO} heartbeat failure detection
\label{sec-5-4-1-4-3-4-2}

\item {\bfseries\sffamily TODO} start a new program container
\label{sec-5-4-1-4-3-4-3}

\item {\bfseries\sffamily TODO} remote start a program container
\label{sec-5-4-1-4-3-4-4}

\item {\bfseries\sffamily TODO} stop a program over MQ
\label{sec-5-4-1-4-3-4-5}

\item {\bfseries\sffamily TODO} remote stop a program over MQ
\label{sec-5-4-1-4-3-4-6}

\item {\bfseries\sffamily TODO} shut down a program container
\label{sec-5-4-1-4-3-4-7}

\item {\bfseries\sffamily TODO} remote shut down a container
\label{sec-5-4-1-4-3-4-8}

\item {\bfseries\sffamily TODO} kill a program
\label{sec-5-4-1-4-3-4-9}

\item {\bfseries\sffamily TODO} remote kill a program
\label{sec-5-4-1-4-3-4-10}

\item {\bfseries\sffamily TODO} STONITH a container
\label{sec-5-4-1-4-3-4-11}
\end{enumerate}
\item Catullus
\label{sec-5-4-1-4-3-5}

Catullus handles the textual interface whereby human-provided strings
are parsed and tokenized into propositions understandable to the AI
characters, and rendering the “thoughts" of AI characters into
string form.

Gaius Valerius Catullus was a noted poet/songwriter.

\begin{enumerate}
\item {\bfseries\sffamily TODO} accept sexp from MQ / tap-n-talk
\label{sec-5-4-1-4-3-5-1}

\item {\bfseries\sffamily TODO} parse incoming utterance properly
\label{sec-5-4-1-4-3-5-2}

\item {\bfseries\sffamily TODO} return utterance in English
\label{sec-5-4-1-4-3-5-3}

\begin{enumerate}
\item {\bfseries\sffamily TODO} proper handling of edge cases of English
\label{sec-5-4-1-4-3-5-3-1}
\end{enumerate}
\item {\bfseries\sffamily TODO} adjudicate utterance
\label{sec-5-4-1-4-3-5-4}

\item {\bfseries\sffamily TODO} evaluate utterance
\label{sec-5-4-1-4-3-5-5}

\item {\bfseries\sffamily TODO} answer yes/no questions
\label{sec-5-4-1-4-3-5-6}

\item {\bfseries\sffamily TODO} anaphora resolution
\label{sec-5-4-1-4-3-5-7}

\item {\bfseries\sffamily TODO} unbound anaphora resolution: ask questions
\label{sec-5-4-1-4-3-5-8}

\item {\bfseries\sffamily TODO} unbound anaphora resolution: take a guess
\label{sec-5-4-1-4-3-5-9}

\item {\bfseries\sffamily TODO} adjudication trust levels / lie, joke
\label{sec-5-4-1-4-3-5-10}

\item {\bfseries\sffamily TODO} failed adjudication: decide joke or lie
\label{sec-5-4-1-4-3-5-11}

\item {\bfseries\sffamily TODO} encode utterance for MQ / reply
\label{sec-5-4-1-4-3-5-12}

\item {\bfseries\sffamily TODO} prompted utterances: self-introduction
\label{sec-5-4-1-4-3-5-13}

\item {\bfseries\sffamily TODO} prompted utterances: curiosity
\label{sec-5-4-1-4-3-5-14}

\item {\bfseries\sffamily TODO} prompted utterances: make a request
\label{sec-5-4-1-4-3-5-15}

\item {\bfseries\sffamily TODO} prompted utterances: imperative
\label{sec-5-4-1-4-3-5-16}

\item {\bfseries\sffamily TODO} prompted utterances: statement of interest
\label{sec-5-4-1-4-3-5-17}

\item {\bfseries\sffamily TODO} imperative handling: adjudicate obedience
\label{sec-5-4-1-4-3-5-18}

\item {\bfseries\sffamily TODO} imperative handling: resolve ambiguities
\label{sec-5-4-1-4-3-5-19}
\end{enumerate}
\item Clodia
\label{sec-5-4-1-4-3-6}

Clodia handles the processing of the artificially intelligent
characters.

Clodia Metelli Pulcher (also spelled Claudia) was the
grand-daughter of Appius Claudius Caecus, and notable for her
political intrigues, enmity with Cicero, disregard for the
admiration of Catullus, and allegations of murdering political
figures by poison.

\begin{enumerate}
\item {\bfseries\sffamily TODO} QoS reporting
\label{sec-5-4-1-4-3-6-1}

\item {\bfseries\sffamily TODO} AI core identity, intelligence, control subroutines
\label{sec-5-4-1-4-3-6-2}

\item {\bfseries\sffamily TODO} core memory model
\label{sec-5-4-1-4-3-6-3}

\begin{enumerate}
\item {\bfseries\sffamily TODO} forgetfulness
\label{sec-5-4-1-4-3-6-3-1}

\item {\bfseries\sffamily TODO} common data core
\label{sec-5-4-1-4-3-6-3-2}

\begin{enumerate}
\item {\bfseries\sffamily TODO} knowledge of physics
\label{sec-5-4-1-4-3-6-3-2-1}

\item {\bfseries\sffamily TODO} knowledge of game world history
\label{sec-5-4-1-4-3-6-3-2-2}

\item {\bfseries\sffamily TODO} knowledge of geography
\label{sec-5-4-1-4-3-6-3-2-3}

\item {\bfseries\sffamily TODO} knowledge of inventory
\label{sec-5-4-1-4-3-6-3-2-4}
\end{enumerate}
\end{enumerate}
\item {\bfseries\sffamily TODO} observation
\label{sec-5-4-1-4-3-6-4}

\begin{enumerate}
\item {\bfseries\sffamily TODO} passive attention
\label{sec-5-4-1-4-3-6-4-1}

\item {\bfseries\sffamily TODO} active attention
\label{sec-5-4-1-4-3-6-4-2}
\end{enumerate}
\item {\bfseries\sffamily TODO} like/dislike system
\label{sec-5-4-1-4-3-6-5}

\item {\bfseries\sffamily TODO} virtue evaluation
\label{sec-5-4-1-4-3-6-6}

\item {\bfseries\sffamily TODO} clothing selection
\label{sec-5-4-1-4-3-6-7}

\item {\bfseries\sffamily TODO} self-preservation instincts
\label{sec-5-4-1-4-3-6-8}

\item {\bfseries\sffamily TODO} navigation
\label{sec-5-4-1-4-3-6-9}

\item {\bfseries\sffamily TODO} conversation
\label{sec-5-4-1-4-3-6-10}

\item {\bfseries\sffamily TODO} crafting
\label{sec-5-4-1-4-3-6-11}

\item {\bfseries\sffamily TODO} game-linked behaviours
\label{sec-5-4-1-4-3-6-12}

\begin{enumerate}
\item {\bfseries\sffamily TODO} playing sport
\label{sec-5-4-1-4-3-6-12-1}

\begin{enumerate}
\item {\bfseries\sffamily TODO} observing and refereeing
\label{sec-5-4-1-4-3-6-12-1-1}

\item {\bfseries\sffamily TODO} playing soccer
\label{sec-5-4-1-4-3-6-12-1-2}

\item {\bfseries\sffamily TODO} playing volleyball
\label{sec-5-4-1-4-3-6-12-1-3}

\item {\bfseries\sffamily TODO} playing baseball
\label{sec-5-4-1-4-3-6-12-1-4}

\item {\bfseries\sffamily TODO} playing basketball
\label{sec-5-4-1-4-3-6-12-1-5}

\item {\bfseries\sffamily TODO} playing “asteroids”
\label{sec-5-4-1-4-3-6-12-1-6}
\end{enumerate}
\item {\bfseries\sffamily TODO} Smudge: making a mess
\label{sec-5-4-1-4-3-6-12-2}

\item {\bfseries\sffamily TODO} Targ: breaking shit
\label{sec-5-4-1-4-3-6-12-3}

\item {\bfseries\sffamily TODO} crafting things
\label{sec-5-4-1-4-3-6-12-4}

\item {\bfseries\sffamily TODO} generating mazes
\label{sec-5-4-1-4-3-6-12-5}

\item {\bfseries\sffamily TODO} pirates: burying treasure
\label{sec-5-4-1-4-3-6-12-6}

\item {\bfseries\sffamily TODO} mail delivery
\label{sec-5-4-1-4-3-6-12-7}

\item {\bfseries\sffamily TODO} mail theft
\label{sec-5-4-1-4-3-6-12-8}

\item {\bfseries\sffamily TODO} construction crew
\label{sec-5-4-1-4-3-6-12-9}

\item {\bfseries\sffamily TODO} singing
\label{sec-5-4-1-4-3-6-12-10}

\item {\bfseries\sffamily TODO} space launch crew
\label{sec-5-4-1-4-3-6-12-11}
\end{enumerate}
\end{enumerate}
\item Galen
\label{sec-5-4-1-4-3-7}

Galen handles the system whereby superposed states of quiesced
arrondissements are collapsed into a discrete state. In other
words: Galen burgeons areas that had been quiesced previously.

Galen was a noted philosopher, logician, and inventor.

This subsystem won't be in place yet for Romance 2.0 
\item Narcissus
\label{sec-5-4-1-4-3-8}

Narcissus handles the simulation of physical forces.

Named for the famous wrestler Narcissus, who may have once
assassinated an emperor, not the mythological character who was
turned into a flower.

\begin{enumerate}
\item {\bfseries\sffamily TODO} Interim basic ground-based non-physics
\label{sec-5-4-1-4-3-8-1}

just enough to get us going, for testing and such, not enough to keep
us running indefinitely.
\item {\bfseries\sffamily TODO} New CL bindings for Bullet
\label{sec-5-4-1-4-3-8-2}

\item {\bfseries\sffamily TODO} world sub-region divisions
\label{sec-5-4-1-4-3-8-3}

how to load-balance physics between Narcissus / Bullet nodes without
having weird freakish edges
\end{enumerate}
\item Lutatius
\label{sec-5-4-1-4-3-9}

Gaius Lutatius Catulus [Latin: C·LVTATIVS·C·F·CATVLVS] was a Roman
statesman and naval commander in the First Punic War.
Temple to Juturna, built by Catulus to celebrate his victory at
Aegades islands, in Largo di Torre Argentina, Rome.

He was elected as a consul in 242 BC, a novus homo. During his
consulship he supervised the construction of a new Roman fleet. This
fleet was funded by donations from wealthy citizens, since the public
treasury was virtually empty. He then led the fleet into victory over
Hanno the Great's Carthaginian fleet in the Battle of the
Aegates Islands. This was the decisive battle of the First Punic
War. To celebrate his victory, he built a temple to Juturna in Campus
Martius, in the area currently known as Largo di Torre Argentina.

\begin{enumerate}
\item {\bfseries\sffamily TODO} inventory equipping
\label{sec-5-4-1-4-3-9-1}

\item {\bfseries\sffamily TODO} using an item
\label{sec-5-4-1-4-3-9-2}

\item {\bfseries\sffamily TODO} sidekick enablement
\label{sec-5-4-1-4-3-9-3}
\end{enumerate}
\item unnamed module / system needed for …
\label{sec-5-4-1-4-3-10}


\begin{enumerate}
\item {\bfseries\sffamily TODO} player registration
\label{sec-5-4-1-4-3-10-1}

\item {\bfseries\sffamily TODO} player passwd
\label{sec-5-4-1-4-3-10-2}

\item {\bfseries\sffamily TODO} player chfn
\label{sec-5-4-1-4-3-10-3}

\item {\bfseries\sffamily TODO} player authentication
\label{sec-5-4-1-4-3-10-4}

\item {\bfseries\sffamily TODO} child account manipulation / sudo-like
\label{sec-5-4-1-4-3-10-5}

\item {\bfseries\sffamily TODO} enforcing parental controls
\label{sec-5-4-1-4-3-10-6}

\item {\bfseries\sffamily TODO} sidekick enablement
\label{sec-5-4-1-4-3-10-7}
\end{enumerate}
\end{enumerate}
\item \texttt{src/static}
\label{sec-5-4-1-4-4}

JavaScript that we blindly copy into place

\begin{enumerate}
\item {\bfseries\sffamily TODO} get upstream libs into the build system
\label{sec-5-4-1-4-4-1}
\end{enumerate}
\item \texttt{src/tools}
\label{sec-5-4-1-4-5}

Homebrew tools used \textbf{during compilation} go in here, e.g. the
parenscript compiler utility should be here;

only the ones that get compiled, stuff like Bash or Perl scripts can
live in \texttt{tools} directly
\item \texttt{src/violetvolts.html}
\label{sec-5-4-1-4-6}

the actual “play now” HTML page

all the HTML overlays should be in here, styled display:none; might
want to break them into separate files/fragments and build the file up
somehow, but for now that doesn't seem to be much of a problem
\end{enumerate}
\item \texttt{tools}
\label{sec-5-4-1-5}

Stuff used to compile that isn't in Stock Fedora. 

Should be Git Submodules where possible or inherited Subversion
repos possibly. 

Google Closure Compiler

YUI compressor

Moved Uglify-JS out, it's in Stock Fedora, yay. (but it's even worse
than YUI? wtf?)

Also: custom scripts and binaries

\begin{enumerate}
\item {\bfseries\sffamily TODO} \texttt{tools/bin} should have all the scripts
\label{sec-5-4-1-5-1}

\item \texttt{tools/bin/smaller}
\label{sec-5-4-1-5-2}

used to check file sizes and return which one is the smallest

misnamed, it accepts multiple files not just two
\item {\bfseries\sffamily TODO} \texttt{tools/bin/parenscript-compile}
\label{sec-5-4-1-5-3}

once I get around to buildapp:ing this should live here
\item \texttt{tools/swf2svg}
\label{sec-5-4-1-5-4}

this is a Perl program that is a work in progress

it rips apart SWF files using \texttt{swfdump} and writes out SVG of all the
artistic components

I just wrote this as a way to see what assets we get as SWF:s from Res
--- so it can be dumped altogether.
\item {\bfseries\sffamily TODO} \texttt{tools/bin/swf-to-3d}\hfill{}\textsc{JUNK}
\label{sec-5-4-1-5-5}

Based on \texttt{swf2svg}, can be dumped.

note that GLGE has an XML format but that's damned verbose
\item {\bfseries\sffamily TODO} fix up the existing Tools like Closure Compiler
\label{sec-5-4-1-5-6}

pull in as submodules or whatever
\end{enumerate}
\end{enumerate}
\subsubsection{{\bfseries\sffamily DONE} \texttt{Makefile}}
\label{sec-5-4-2}

The Makefile is pretty solid now, at least for the code parts.
\subsubsection{{\bfseries\sffamily TODO} Parenscript compilation}
\label{sec-5-4-3}

Parenscript gets routed via a Lisp program to handle some odd cases.

Basically:

\begin{itemize}
\item First, we load macros out of 00-macros.lisp

\item We compile each file form-by-form

\item we concatenate the results onto 01-preamble.js — where we put any
other snippets that haven't made it into Parenscript yet
\end{itemize}

\begin{enumerate}
\item {\bfseries\sffamily TODO} source maps
\label{sec-5-4-3-1}
\end{enumerate}
\subsubsection{{\bfseries\sffamily DONE} Minifiers for JS, CSS, HTML}
\label{sec-5-4-4}

I have a few minifiers in place, \texttt{make} basically “races” them
against one another and the smallest output file “wins,” which has a
downside that the build might be minified differently from day to day

\begin{enumerate}
\item {\bfseries\sffamily TODO} maybe just Closure
\label{sec-5-4-4-1}

I'm seeing such a massive difference/improvement using Closure that I
might want to just use it
\item {\bfseries\sffamily TODO} static libs separately?
\label{sec-5-4-4-2}

The various static JS libs might do better being minified their own
ways or loaded from a CDN

particularly they might not cope well with the ADVANCED mode of
Closure Compiler so we might want to minify them with a gentler
setting and then use ADVANCED for our own code, where it'll be easier
to fix it when it breaks (e.g. adding JSDoc annotations or being
careful about using dot-accessors versus string-accessors)
\end{enumerate}
\section{Policies}
\label{sec-6}
\section{Non-Violence Manifesto}
\label{sec-7}

\subsection{The original notes from 2009}
\label{sec-7-1}

• Darkstar Forces II: Jedi Knight

• God of War

• Castlevania: Harmony of Dissonance

• Avernum

• The Secret of Monkey Island series

• Phoenix Wright: Ace Attorney

• The Seventh Guest, The Eleventh Hour

• Blue Dragon

• The Journeyman Project

• American Mc Gee’s Alice

• The White Chamber

• Shivers

• Killer7

• Silent Hill

• Hugo’s House of Horrors

• Myst series

• Sokoban

• Kwirk

• Tomb Raider series

• Pokemon series

• Paper Mario

• Lufia II

• Golden Sun series

• Little Big Adventure

• Wild ARMs

• Rogue Galaxy

• Broken Sword 3

• Final Fantasy Mystic Quest

• Tales of Symphonia

• StarFox Adventures

• Perplex City

• La Mulana

• Dragon Warrior III

• Zork series

• Boxxle
% Emacs 24.3.1 (Org mode 8.2.3c)
\end{document}